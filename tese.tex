% ---------------------------------------------------------------------
% --- Arquivo principal e os demais serão os dos capítulos.
% --- EXPRESSÕES ENTRE <> DEVERÃO SER COMPLETADAS COM A INFORMAÇÃO ESPECÍFICA DO TRABALHO 
% ---------------------------------------------------------------------

\documentclass[ruledheader]{abnt_UFF}

%---pacotes para hiphenizacao e acentuacao em portugues
\usepackage[brazil]{babel}
\usepackage[latin1]{inputenc}
\usepackage[T1]{fontenc}

%--- pacote para figuras
\usepackage{epsf}
\usepackage[dvips]{epsfig,graphicx}
\usepackage{subfigure}

%--- pacote de simbolos
\usepackage{latexsym}
\usepackage{textcomp}

%--- simbolos matematicos
\usepackage{amsmath}
\usepackage{amssymb}

%--- pacote para gerar pseudo-codigo
\usepackage{algorithm}
\usepackage{algorithmic}
\floatname{algorithm}{Algoritmo}

%--- outros pacotes
\usepackage{url}
\usepackage{longtable}
\usepackage{lscape}
\usepackage{pdfpages} % pacote para inserir ficha catalográfica

%--- tabela colorida
\usepackage{colortbl}
\usepackage{multicol}
\usepackage{multirow}
\usepackage{rotating}


\hyphenation{
a-de-qua-da-men-te 
di-men-sio-na-men-to 
}

%---------usando tipo de fonte padrao
\renewcommand{\ABNTchapterfont}{\bfseries\fontfamily{cmr}\fontseries{b}\selectfont}
\renewcommand{\ABNTsectionfont}{\bfseries\fontfamily{cmr}}

\usepackage[pdftex]{hyperref}
\hypersetup{colorlinks,
citecolor=black,
linkcolor=black,
urlcolor=black,
%allbordercolors=white,
}

%--- desenhar diagramas bloco de controle
\usepackage{schemabloc}
\usetikzlibrary{circuits}

%--- bibliografia abntex
\usepackage[brazilian,hyperpageref]{backref}	% Páginas com as citações na bibliografia
\usepackage[num,	% Padrão de citação numérico
abnt-etal-list=0,	% Colocar todos os autores nas referencias
abnt-etal-cite=2	% Colocar et al. nas citações com mais de 2 autores
abnt-doi=doi,	% Padrão de doi na bibliografia
%bibjustif,      % Bibliografia justificada
]{abntex2cite}	% Citações padrão ABNT

% --- 
% CONFIGURAÇÕES DE PACOTES
% --- 

% ---
% Configurações do pacote abntex2cite
\citebrackets[]	% demarcador de numeros

% ---
% Configurações do pacote backref
% Usado sem a opção hyperpageref de backref
\renewcommand{\backrefpagesname}{Citado na(s) p{\' a}gina(s):~}
% Texto padrão antes do número das páginas
\renewcommand{\backref}{}
% Define os textos da citação
\renewcommand*{\backrefalt}[4]{
	\ifcase #1 %
	Nenhuma cita{\c c}{\~a}o no texto.%
	\or
	Citado na p{\'a}gina #2.%
	\else
	Citado #1 vezes nas p{\'a}ginas #2.%
	\fi}%
% ---

% --- -----------------------------------------------------------------
% --- Documento Principal.
% --- -----------------------------------------------------------------

\begin{document}

% --- -----------------------------------------------------------------
% --- Titulo, abstract, dedicatórias e agradecimentos.
% --- Índice geral, lista de figuras e tabelas.
% --- -----------------------------------------------------------------
% --- -----------------------------------------------------------------
% --- Elementos usados na Capa e na Folha de Rosto.
% --- EXPRESSÕES ENTRE <> DEVERÃO SER COMPLETADAS COM A INFORMAÇÃO ESPECÍFICA DO TRABALHO
% --- E OS SÍMBOLOS <> DEVEM SER RETIRADOS
% --- -----------------------------------------------------------------
\autor{NO{\'E} DE LIMA BEZERRA} % deve ser escrito em maiúsculo

\titulo{SIMULADOR PARA REJEI{\c C}{\~A}O OTIMIZADA DE CARGA EM PLANTAS INDUSTRIAIS}

\instituicao{UNIVERSIDADE FEDERAL FLUMINENSE}

\orientador{JULIO CESAR STACCHINI DE SOUZA}

\coorientador{MILTON BROWN DO COUTTO FILHO} % se não existir co-orientador apague essa linha

\local{NITER{\'O}I}

\data{2019} % ano da defesa

\comentario{Disserta{\c c}{\~a}o de Mestrado apresentada ao Programa de P{\'o}s-Gradua{\c c}{\~a}o em Computa{\c c}{\~a}o da \mbox{Universidade} Federal Fluminense como requisito parcial para a obten{\c c}{\~a}o do Grau de \mbox{Mestre em Computa{\c c}{\~a}o}. {\'A}rea de concentra{\c c}{\~a}o: COMPUTA{\c C}{\~A}O CIENT{\'I}FICA E SISTEMAS DE POT{\^E}NCIA} %preencha com a sua area de concentracao


% --- -----------------------------------------------------------------
% --- Capa. (Capa externa, aquela com as letrinhas douradas)(Obrigatório)
% --- ----------------------------------------------------------------
\capa

% --- -----------------------------------------------------------------
% --- Folha de rosto. (Obrigatório)
% --- ----------------------------------------------------------------
\folhaderosto

\pagestyle{ruledheader}
\setcounter{page}{1}
\pagenumbering{roman}

% --- -----------------------------------------------------------------
% --- Ficha Catalográfica. (Obrigatório)
% --- ----------------------------------------------------------------
\includepdf[pages=-]{ficha_catalografica.pdf}
%\cleardoublepage
%\thispagestyle{empty}

%\vspace*{120mm}

%\includegraphics[width=0.9\linewidth]{figuras/ficha}


% --- -----------------------------------------------------------------
% --- Termo de aprovação. (Obrigatório)
% --- ----------------------------------------------------------------
\includepdf[pages=-]{aprovacao.pdf}

% Original substituído por versão escaneada
%\cleardoublepage
%\thispagestyle{empty}

%\vspace{-60mm}

%\begin{center}
%   {\large NO{\'E} DE LIMA BEZERRA}\\
%   \vspace{7mm}

%   SIMULADOR PARA REJEI{\c C}{\~A}O OTIMIZADA DE CARGA EM PLANTAS INDUSTRIAIS\\
%  \vspace{10mm}
%\end{center}

%\noindent
%\begin{flushright}
%\begin{minipage}[t]{8cm}

%Disserta{\c c}{\~a}o de Mestrado apresentada ao Programa de P{\'o}s-Gradua{\c c}{\~a}o em Computa{\c c}{\~a}o da Universidade Federal Fluminense como requisito parcial para a obten{\c c}{\~a}o do \mbox{Grau} de Mestre em Computa{\c c}{\~a}o. {\'A}rea de concentra{\c c}{\~a}o: COMPUTA{\c C}{\~A}O CIENT{\'I}FICA E SISTEMAS DE POT{\^E}NCIA %preencha com a sua area de concentracao

%\end{minipage}
%\end{flushright}
%\vspace{5mm}
%\noindent
%Aprovada em 18 de novembro de 2019.
%\begin{flushright}
%  \parbox{11cm}
%  {
%  \begin{center}
%  BANCA EXAMINADORA \\
%  \vspace{6mm}
%  \rule{11cm}{.1mm} \\
%    Prof. JULIO CESAR STACCHINI DE SOUZA - Orientador, IC/UFF \\
%    \vspace{6mm}
%  \rule{11cm}{.1mm} \\
%    Prof. MILTON BROWN DO COUTTO FILHO - Orientador, IC/UFF \\
%    \vspace{6mm}
%  \rule{11cm}{.1mm} \\
%    Prof. RICARDO LEIDERMAN, IC/UFF \\
%  \vspace{6mm}
%  \rule{11cm}{.1mm} \\
%    Prof. GLAUCO NERY TARANTO, COPPE/UFRJ \\
%    \vspace{6mm}
%  \end{center}
%  }
%\end{flushright}
%\begin{center}
  %\vspace{2mm}
%  Niter{\'o}i \\
  %\vspace{6mm}
%  2019

%\end{center}

% --- -----------------------------------------------------------------
% --- Dedicatória.(Opcional)
% --- -----------------------------------------------------------------
\cleardoublepage
\thispagestyle{empty}
\vspace*{200mm}

\begin{flushright}
{\em 
Aos meus professores, que t{\~a}o generosamente colaboram com a continuidade e amplia{\c c}{\~a}o do conhecimento.
}
\end{flushright}
\newpage


% --- -----------------------------------------------------------------
% --- Agradecimentos.(Opcional)
% --- -----------------------------------------------------------------
\pretextualchapter{Agradecimentos}
\hspace{5mm}
{\`A} minha fam{\'\i}lia, em especial {\`a} minha esposa Camila Borduam, pelo apoio e compreens{\~a}o quanto {\`a}s noites de sono suprimidas em dedica{\c c}{\~a}o aos estudos.

Aos meus professores que, de forma generosa e altru{\'\i}sta, compartilharam comigo uma parcela de conhecimento e experi{\^e}ncia.

Aos meus orientadores, professores Milton Brown e Julio Stacchini, que pacientemente me guiaram e inflaram {\^a}nimo quando necess{\'a}rio.

Ao professor Luiz Satoru Ochi que, al{\'e}m de mostrar-se um grande amigo, providenciou todos os recursos que me foram necess{\'a}rios, bem como ensinou tudo o que aprendi sobre metaheur{\'\i}stica, grande pilar neste trabalho.

Aos meus colegas de trabalho, que colaboraram na forma{\c c}{\~a}o da minha experi{\^e}ncia profissional, bem como tornaram vi{\'a}vel a concilia{\c c}{\~a}o dos estudos com o trabalho.

Ao grande amigo que a vida me presenteou e tanto me salvou na elabora{\c c}{\~a}o desta Disserta{\c c}{\~a}o, Joubert Gon{\c c}alves.

% --- -----------------------------------------------------------------
% --- Resumo em português.(Obrigatório)
% --- -----------------------------------------------------------------
\begin{resumo}

Plantas industriais que disp{\~o}em de gera{\c c}{\~a}o pr{\'o}pria de energia el{\'e}trica podem enfrentar diversos problemas ao perder uma unidade geradora, at{\'e} mesmo aqueles da perda total do sistema el{\'e}trico. Objetivando minimizar os preju{\'\i}zos decorrentes de um completo desligamento da planta, usualmente, um sistema de rejei{\c c}{\~a}o de cargas entra em a{\c c}{\~a}o, desligando cargas selecionadas de modo a evitar a queda do restante do sistema. Este trabalho apresenta uma metodologia que busca selecionar de forma otimizada as cargas a serem desligadas, de forma a minimizar o impacto total deste desligamento. A metodologia utilizada combina um modelo matem{\'a}tico para a avalia{\c c}{\~a}o de cada solu{\c c}{\~a}o candidata e um m{\'e}todo heur{\'\i}stico de busca computacional para resolver o problema da minimiza{\c c}{\~a}o do custo operacional da planta ap{\'o}s a ocorr{\^e}ncia de um evento. Neste trabalho foi desenvolvido um simulador computacional no qual podem ser representadas as condi{\c c}{\~o}es de opera{\c c}{\~a}o da planta, a ocorr{\^e}ncia de eventos, sendo tamb{\'e}m fornecida a solu{\c c}{\~a}o otimizada para o descarte de cargas de acordo com a proposta. Tal simulador pode ser utilizado como ferramenta de suporte {\`a} tomada de decis{\~a}o, sendo tamb{\'e}m um primeiro passo na dire{\c c}{\~a}o da automa{\c c}{\~a}o das decis{\~o}es sobre rejei{\c c}{\~a}o de carga em plantas industriais. Diversos testes foram realizados para fins de valida{\c c}{\~a}o da metodologia proposta e utiliza{\c c}{\~a}o do simulador computacional desenvolvido, tendo sido caracterizados diferentes eventos em uma planta industrial adaptada a partir de um caso real. Os resultados obtidos mostram a utilidade do simulador computacional constru{\'\i}do e a efetividade da metodologia proposta na obten{\c c}{\~a}o de uma solu{\c c}{\~a}o otimizada para o problema.

{\hspace{-8mm} \bf{Palavras-chave}}: Rejei{\c c}{\~a}o de carga; descarte de carga; meta-heur{\'\i}stica; intelig{\^e}ncia computacional.

\end{resumo}

% --- -----------------------------------------------------------------
% --- Resumo em língua estrangeira.(Obrigatório)
% --- -----------------------------------------------------------------
\begin{abstract}

Industrial plants that have their own power generation can face several problems when losing a generating unit, even the critical scenario of a total loss of the electrical system. In order to minimize damages resulting from a complete shutdown of the plant, a load shedding system usually takes place, shutting down selected loads to prevent the system collapse. This Dissertation presents a methodology that seeks to optimize the process of selecting the loads to be shed, minimizing the total impact of this shutdown. The methodology proposed combines a mathematical model for the evaluation of each candidate solution with a heuristic computational search method to solve the problem of minimizing the operational cost of the plant after the occurrence of an event. A computer simulator is developed, in which the plant operating conditions are represented, as well as the occurrence of events involving the plant components. It also provides the optimized solution for the load shedding after the application of the proposed methodology. Such a simulator can be used as a decision support tool, being a first step towards the automation of decisions regarding the load rejection in industrial plants. Several tests were performed for the validation of the proposed methodology and the use of the developed computer simulator, under different operating conditions in an industrial plant adapted from a real case. The results show the usefulness of the simulator and the effectiveness of the proposed methodology in obtaining an optimized solution to the problem.

{\hspace{-8mm} \bf{Keywords}}: Load rejection; load shedding; metaheuristic; computational intelligence.

\end{abstract}

% --- -----------------------------------------------------------------
% --- Lista de figuras.(Opcional)
% --- -----------------------------------------------------------------
\cleardoublepage
\listoffigures


% --- -----------------------------------------------------------------
% --- Lista de tabelas.(Opcional)
% --- -----------------------------------------------------------------
\cleardoublepage
\label{pag:last_page_introduction}
\listoftables
\cleardoublepage

% --- -----------------------------------------------------------------
% --- Lista de abreviatura.(Opcional)
%Elemento opcional, que consiste na relação alfabética das abreviaturas e siglas utilizadas no texto, seguidas das %palavras ou expressões correspondentes grafadas por extenso. Recomenda-se a elaboração de lista própria para cada %tipo (ABNT, 2005).
% --- ----------------------------------------------------------------
\cleardoublepage
\pretextualchapter{Lista de Abreviaturas e Siglas}
\begin{tabular}{lcl}
LSS & : & \textit{Load Shedding Schedule}, Tabela de Rejei{\c c}{\~a}o de Cargas;\\
$pu$ & : & Sistema ``\textit{per unit}'', grandeza dividida por um valor de refer{\^e}ncia;\\
UFLS & : & \textit{Under Frequence Load Shedding}, Rejei{\c c}{\~a}o de Cargas por Sub-Frequ{\^e}ncia;\\
UVLS & : & \textit{Under Voltage Load Shedding}, Rejei{\c c}{\~a}o de Cargas por Sub-Tens{\~a}o;\\
ILS & : & \textit{Intelligent Load Shedding}, Rejei{\c c}{\~a}o Inteligente de Cargas;\\
ANN & : & \textit{Artificial Neural Network}, Redes Neurais Artificiais; \\
ERAC & : & Esquema Regional de Al{\'\i}vio de Carga por Sub-Frequ{\^e}ncia \\
VND & : & \textit{Variable Neighborhood Descent}, Varia{\c c}{\~a}o Descendente de Vizinhan{\c c}a; \\
GUI & : & \textit{Graphical User Interface}, Interface Gr{\'a}fica de Usu{\'a}rio; \\
API & : & \textit{Application Programming Interface}, Interface de Programa{\c c}{\~a}o; \\
CCM & : & Centro de Controle de Motor. \\
\end{tabular}

% --- -----------------------------------------------------------------
% --- Sumario.(Obrigatorio)
% --- -----------------------------------------------------------------
\pagestyle{ruledheader}
\tableofcontents




% --- -----------------------------------------------------------------
% --- Insercao dos capitulos.
% --- -----------------------------------------------------------------
\pagestyle{ruledheader}
\setcounter{page}{1}
\pagenumbering{arabic}
\chapter{Introdu{\c c}{\~a}o} \label{cap:intro}

\section{Motiva{\c c}{\~a}o} \label{sec:carac}

Sistemas el{\'e}tricos de pot{\^e}ncia est{\~a}o, geralmente, particionados na gera{\c c}{\~a}o, transmiss{\~a}o e distribui{\c c}{\~a}o de energia el{\'e}trica. Requisitos de qualidade definem os n{\'\i}veis de tens{\~a}o e frequ{\^e}ncia da rede el{\'e}trica, bem como seus limites de toler{\^a}ncia. Em um sistema equilibrado, a pot{\^e}ncia consumida pelas cargas {\'e} correspondente {\`a} gerada (diferindo apenas pelas perdas inerentes a qualquer sistema). Entretanto, frente a eventos, a frequ{\^e}ncia sofre perturba{\c c}{\~o}es. Geralmente, oscila{\c c}{\~o}es ocorrem quando h{\'a} s{\'u}bita entrada ou sa{\'\i}da de cargas significativas na rede, bem como sa{\'\i}da de unidades geradoras. Essas varia{\c c}{\~o}es de frequ{\^e}ncia devem ser absorvidas e compensadas pelos demais geradores, atrav{\'e}s do controle prim{\'a}rio de gera{\c c}{\~a}o \cite{1597614,6811137,6689316,6085887}. Entretanto, sistemas ilhados de gera{\c c}{\~a}o ou cogera{\c c}{\~a}o (plantas industriais com gera{\c c}{\~a}o pr{\'o}pria e carga, geralmente sem transmiss{\~a}o, eventualmente ligados ao sistema nacional em caso de cogera{\c c}{\~a}o, ou totalmente isolados, como os ilhados) podem, eventualmente, ter capacidade de gera{\c c}{\~a}o remanescente inferior {\`a} carga total ap{\'o}s a perturba{\c c}{\~a}o. Para que o sistema possa se recuperar sem resultar em um blecaute {\'e} necess{\'a}rio o desligamento de cargas el{\'e}tricas emergencialmente. Esse processo {\'e} chamado de Descarte, Al{\'\i}vio ou Rejei{\c c}{\~a}o de Carga\footnote{Ou, em ingl{\^e}s, \textit{Load Shedding}.} e obedece a uma ordem pr{\'e} definida, cuja determina{\c c}{\~a}o ser{\'a} objeto de estudo neste trabalho.

Para estabelecer o balan{\c c}o de pot{\^e}ncia gera{\c c}{\~a}o-carga, alguns aspectos b{\'a}sicos devem ser observados pelos esquemas de rejei{\c c}{\~a}o de carga:

\begin{itemize}
    \item evitar atua{\c c}{\~o}es indevidas no sistema el{\'e}trico;
    \item atuar rapidamente;
    \item minimizar o n{\'u}mero de desligamentos (montante) de cargas;
    \item respeitar as prioridades estabelecidas para cada carga.
\end{itemize}

Plantas industriais de facilidades el{\'e}tricas t{\^e}m demandas de energia el{\'e}trica diferenciadas, pois apresentam alto consumo energ{\'e}tico em seus processos e, ao mesmo tempo, requerem qualidade e confiabilidade no fornecimento desta energia. Uma interrup{\c c}{\~a}o deste suprimento pode causar preju{\'\i}zos por vezes inestim{\'a}veis, pois, al{\'e}m da perda de produ{\c c}{\~a}o inerente {\`a} parada do processo, pode comprometer a vida {\'u}til dos equipamentos (os quais t{\^e}m alto custo de reparo ou substitui{\c c}{\~a}o), ou ainda, a seguran{\c c}a das pessoas envolvidas, causando acidentes.

Como os custos de interrup{\c c}{\~a}o j{\'a} s{\~a}o elevados por natureza, o risco intr{\'\i}nseco justifica a instala{\c c}{\~a}o de sistemas de gera{\c c}{\~a}o particular, de forma a atender apenas a planta em quest{\~a}o. Desta forma, a concession{\'a}ria que distribui energia no local do parque industrial pode ou n{\~a}o ser o principal fornecedor de energia, mas caso esta sofra uma interrup{\c c}{\~a}o, o sistema local assume as cargas imediatamente.

Outro aspecto que leva algumas ind{\'u}strias a utilizarem sistemas pr{\'o}prios {\'e} a diferen{\c c}a no pre{\c c}o da energia nos hor{\'a}rios de ponta\footnote{Faixa hor{\'a}rio definida como maior demanda da rede, elevando o pre{\c c}o da energia.}, bem como as varia{\c c}{\~o}es no fator de pot{\^e}ncia el{\'e}trico. A t{\'\i}tulo de exemplo, {\'e} comum instala{\c c}{\~o}es de \textit{Shoppings Centers} ou Supermercados disporem de geradores el{\'e}tricos para utiliz{\'a}-los nos hor{\'a}rios de ponta.

Entretanto, h{\'a} casos particulares que imp{\~o}em por si a necessidade de utiliza{\c c}{\~a}o de geradores, geralmente v{\'a}rios, sem a op{\c c}{\~a}o de fornecimento da concession{\'a}ria, como por exemplo os sistemas de avia{\c c}{\~a}o e embarca{\c c}{\~o}es.

Na ocorr{\^e}ncia de um evento que comprometa a gera{\c c}{\~a}o de energia el{\'e}trica, comumente surge uma varia{\c c}{\~a}o de frequ{\^e}ncia no sistema, diretamente proporcional {\`a} intensidade do dist{\'u}rbio ocorrido. Ocorrendo uma diminui{\c c}{\~a}o da gera{\c c}{\~a}o, para atender a carga, a frequ{\^e}ncia sofre uma diminui{\c c}{\~a}o. Em sentido oposto, quando h{\'a} um corte de carga no sistema, observa-se uma tend{\^e}ncia de crescimento da frequ{\^e}ncia. Tais altera{\c c}{\~o}es de frequ{\^e}ncia s{\~a}o sentidas por rel{\'e}s de frequ{\^e}ncia \cite{mamede2000proteccao} e controladores l{\'o}gicos program{\'a}veis que atuam para o reestabelecimento do equil{\'i}brio entre carga e gera{\c c}{\~a}o.

Esquemas de rejei{\c c}{\~a}o de carga atuam apenas ap{\'o}s geradores remanescentes perderem parte da sua in{\'e}rcia nominal e apresentarem certa varia{\c c}{\~a}o de velocidade, o que em muitos casos acaba comprometendo a estabilidade ou provocando a necessidade de um descarte adicional de carga (na maioria dos casos), para que os geradores retornem a rota{\c c}{\~a}o nominal e a frequ{\^e}ncia seja normalizada.

Al{\'e}m disto, tais sistemas baseiam-se em procedimentos em que um n{\'u}mero fixo de cargas s{\~a}o desligadas, consoante prioridades previamente estabelecidas, o que pode levar a um descarte de carga superior ao necess{\'a}rio para o equil{\'\i}brio gera{\c c}{\~a}o-carga.

Alternativamente, sistemas de rejei{\c c}{\~a}o de carga adaptativos desligam apenas o montante de carga necess{\'a}rio ao reestabelecimento do equil{\'\i}brio do sistema, correspondente {\`a} quantidade de pot{\^e}ncia que deixou de ser suprida. Neste esquema, torna-se necess{\'a}rio considerar continuamente as condi{\c c}{\~o}es operacionais do sistema el{\'e}trico, obtidas preferencialmente por dispositivos de medi{\c c}{\~a}o. Os resultados obtidos s{\~a}o armazenados em uma tabela din{\^a}mica, que define as cargas a serem rejeitadas.

\section{Objetivos}\label{sec:prop}

Visando obter uma solu{\c c}{\~a}o otimizada para o problema de rejei{\c c}{\~a}o de carga, este trabalho prop{\~o}e a aplica{\c c}{\~a}o de t{\'e}cnicas de intelig{\^e}ncia computacional aliadas a m{\'e}todos de atua{\c c}{\~a}o existentes em rel{\'e}s (utilizando a fun{\c c}{\~a}o 81 \cite[Tabela~1.4 e Se{\c c}{\~a}o~3.9]{mamede2000proteccao}). Assim, uma perturba{\c c}{\~a}o na frequ{\^e}ncia el{\'e}trica da rede, bem como o desligamento inesperado de um gerador, disparam a atua{\c c}{\~a}o de um rel{\'e} que efetuar{\'a} a rejei{\c c}{\~a}o de cargas em etapas sucessivas. Como ser{\'a} visto no Cap{\'\i}tulo~\ref{cap:revbib}, um atuador por frequ{\^e}ncia, cuja aplica{\c c}{\~a}o e funcionamento j{\'a} s{\~a}o conhecidos e estabelecidos, ao detectar uma anomalia, far{\'a} desligamentos, buscando os elementos em uma tabela previamente obtida com a metodologia proposta, desligando-os na ordem estabelecida nesta tabela.

Cabe dizer que, na inviabilidade de se realizar testes em sistemas reais, uma simula{\c c}{\~a}o computacional se encarregar{\'a} de fornecer dados estat{\'i}sticos para avaliar o funcionamento do esquema de rejei{\c c}{\~a}o proposto.

\section{Contribui{\c c}{\~o}es}\label{sec:contrib}

Esta Disserta{\c c}{\~a}o prop{\~o}e um tratamento alternativo aos atualmente existentes para a solu{\c c}{\~a}o do problema de restaura{\c c}{\~a}o da condi{\c c}{\~a}o operativa de sistemas el{\'e}tricos com gera{\c c}{\~a}o pr{\'o}pria, ap{\'o}s a ocorr{\^e}ncia da perda em uma unidade geradora ou, ainda, uma sobrecarga transit{\'o}ria causada pela entrada em opera{\c c}{\~a}o de uma carga de grande porte.

Deve-se destacar que as solu{\c c}{\~o}es existentes para o problema em quest{\~a}o s{\~a}o bastante conhecidas e consolidadas do ponto de vista el{\'e}trico e, eventualmente, econ{\^o}mico, no caso das concession{\'a}rias de distribui{\c c}{\~a}o. Entretanto, este trabalho visa combinar aspectos distintos para fornecer uma solu{\c c}{\~a}o diferenciada com m{\'u}ltiplos objetivos, aliando aspectos el{\'e}tricos e operacionais, permitindo ao operador do sistema definir o quanto a decis{\~a}o deve pender para um aspecto ou outro atrav{\'e}s de par{\^a}metros de configura{\c c}{\~a}o do esquema proposto.

Assim, o trabalho contribui para oferecer uma variedade de solu{\c c}{\~o}es que re{\'u}ne aspectos interessantes encontrados separadamente em outros esquemas de rejei{\c c}{\~a}o de carga, sem contudo perder a robustez computacional.

\section{Estrutura}\label{sec:estrut}

Neste Cap{\'\i}tulo foi feita uma descri{\c c}{\~a}o geral do problema em an{\'a}lise e apresentadas as contribui{\c c}{\~o}es deste trabalho de pesquisa. No Cap{\'\i}tulo~\ref{cap:revbib} {\'e} apresentada uma revis{\~a}o da literatura correlata ao tema desta Disserta{\c c}{\~a}o, incluindo o desenvolvimento matem{\'a}tico dos princ{\'\i}pios eletromec{\^a}nicos envolvidos e algumas solu{\c c}{\~o}es e aplica{\c c}{\~o}es existentes. O Cap{\'\i}tulo~\ref{cap:metod} discorrer{\'a} em detalhes sobre a solu{\c c}{\~a}o aqui proposta e o Cap{\'\i}tulo~\ref{cap:impl} apresenta a forma utilizada para demonstra{\c c}{\~a}o e teste desta solu{\c c}{\~a}o. A seguir, o Cap{\'\i}tulo~\ref{cap:teste} apresenta os resultados obtidos atrav{\'e}s de simula{\c c}{\~a}o computacional. Finalmente, o Cap{\'\i}tulo~\ref{cap:concl} apresenta as conclus{\~o}es alcan{\c c}adas no trabalho de pesquisa, bem como indica sugest{\~o}es para continua{\c c}{\~a}o e expans{\~a}o desta Disserta{\c c}{\~a}o.

No Ap{\^e}ndice~\ref{apend:estr}, apresenta-se a estrutura de classes do \textit{software} desenvolvido e uma breve explica{\c c}{\~a}o da formula{\c c}{\~a}o computacional do simulador constru{\'\i}do.

\chapter{Aspectos B{\'a}sicos}\label{cap:revbib}

A varia{\c c}{\~a}o na frequ{\^e}ncia el{\'e}trica da rede traz enormes preju{\'\i}zos, podendo causar defeitos e prejudicar o funcionamento de equipamentos, tais como, motores, transformadores e dispositivos eletr{\^o}nicos. Entretanto, os geradores s{\~a}o os mais prejudicados. A Tabela~\ref{tab:undf} apresenta alguns valores de refer{\^e}ncia para o tempo de subfrequ{\^e}ncia tolerado sobre geradores de pot{\^e}ncia, ressaltando que esses tempos s{\~a}o cumulativos ao longo de toda a vida {\'u}til do equipamento.

\begin{table}[!h]
	\begin{center}
		\caption[Efeito da subfrequ{\^e}ncia sobre um gerador]{Efeito da subfrequ{\^e}ncia sobre um gerador [Fonte: \citeauthor{get6449}, adaptado]}
		\label{tab:undf}
		\vspace{5pt}
		\begin{tabular}{c c}
			\hline
			\textbf{\textbf{Frequ{\^e}ncia a}} & \textbf{Tempo M{\'\i}nimo}\\
			\textbf{\textbf{Plena Carga}} & \textbf{Para Dano}\\
			\hline\hline
			$59,4~Hz$ & cont{\'\i}nuo \\
			$58,8~Hz$ & $90$ minutos \\
			$58,2~Hz$ & $10$ minutos \\
			$57,6~Hz$ & $1$ minuto \\
			\hline
		\end{tabular}
	\end{center}
\end{table}

Portanto, a primeira abordagem para definir quando deve-se iniciar a rejei{\c c}{\~a}o de cargas {\'e} analisar o efeito din{\^a}mico que ocorre entre as frequ{\^e}ncias el{\'e}tricas e mec{\^a}nicas, tanto dos geradores quanto das cargas. A Se{\c c}{\~a}o~\ref{sec:teo} apresenta a influ{\^e}ncia desses efeitos e a forma de atua{\c c}{\~a}o subsequente.

\section{Desenvolvimento Te{\'o}rico} \label{sec:teo}

Da mec{\^a}nica cl{\'a}ssica \cite{umans2013fitzgerald}, a pot{\^e}ncia mec{\^a}nica do gerador, em condi{\c c}{\~o}es ideais, {\'e} dada por:

\begin{equation}
	\label{eq:ptom}
	P_{m} = \tau \cdot \omega
\end{equation}

Onde,

\begin{itemize}
	\item[] $P_{m}$ {\'e} a pot{\^e}ncia mec{\^a}nica de um equipamento rotativo em $W$ ($watt$);
	\item[] $\tau$ {\'e} o torque mec{\^a}nico no eixo em $N\cdot m$ ($newton \times metro$);
	\item[] $\omega$ {\'e} a frequ{\^e}ncia angular em $rad/s$ ($radianos~por~segundo$).
\end{itemize}

Um desequil{\'\i}brio entre gera{\c c}{\~a}o e carga em equipamentos rotativos, em que o torque mec{\^a}nico {\'e} diferente do torque el{\'e}trico, tem-se um torque de acelera{\c c}{\~a}o resultante, de acordo com:

\begin{equation}
	\label{eq:torq}
	T_{G} - T_{L} = T_{a}
\end{equation}

Onde,

\begin{itemize}
	\item[] $T_{G}$ {\'e} o torque mec{\^a}nico em $N \cdot m$;
	\item[] $T_{L}$ {\'e} o torque el{\'e}trico em $N \cdot m$;
	\item[] $T_{a}$ {\'e} o torque de acelera{\c c}{\~a}o da rede em $N \cdot m$.
\end{itemize}

Este desequil{\'i}brio gera uma acelera{\c c}{\~a}o na frequ{\^e}ncia da rede, pois o torque resultante causa uma acelera{\c c}{\~a}o proporcional ao momento de in{\'e}rcia do sistema.

\begin{equation}
	\label{eq:acel}
	J\frac{d\omega_{L}}{dt} = T_{a}
\end{equation}

Onde,

\begin{itemize}
	\item[] $J$ {\'e} o momento de in{\'e}rcia do sistema, em $kg \cdot m^{2}$, dado pela Equa{\c c}{\~a}o~\ref{eq:jinerc};
	\item[] $\omega_{L}$ {\'e} a frequ{\^e}ncia el{\'e}trica do sistema em $rad/s$.
\end{itemize}

\begin{equation}
	\label{eq:jinerc}
	J = \int{r^{2}}{dm}
\end{equation}

Onde,

\begin{itemize}
	\item[] $r$ {\'e} dist{\^a}ncia de cada ponto integrado ao eixo de rota{\c c}{\~a}o (raio) em $m$ ($metro$);
	\item[] $dm$ {\'e} o diferencial de massa em $kg$ ($kilograma$).
\end{itemize}

A Equa{\c c}{\~a}o~\ref{eq:jinerc} {\'e} uma propriedade geom{\'e}trica que expressa a distribui{\c c}{\~a}o da massa dos rotores em torno do eixo rotativo no conjunto eletromec{\^a}nico dos geradores. O momento de in{\'e}rcia representa, para o movimento rotativo, o mesmo que a massa representa para o movimento linear.

Entretanto, por uma quest{\~a}o de conveni{\^e}ncia, ser{\'a} utilizado a constante de in{\'e}rcia $H$, definida na Equa{\c c}{\~a}o~\ref{eq:hinerc}, que representa o momento de in{\'e}rcia em $pu$, na base de pot{\^e}ncia do sistema.

\begin{equation}
	\label{eq:hinerc}
	H = \frac{1}{2}\frac{J\omega_{0_{G}}^{2}}{VA_{base}}
\end{equation}

Assim:

\begin{equation}
	\label{eq:hpj}
	J = \frac{2H}{\omega_{0_{G}}^{2}}\cdot VA_{base}
\end{equation}

Substituindo (\ref{eq:hpj}) em (\ref{eq:acel}), vem:

\begin{equation}
	\label{eq:inerc}
	\frac{H}{\pi f_{0}} \frac{d^{2}\delta}{dt^{2}} = T_{G} - T_{L} = T_{a}
\end{equation}

Onde,

\begin{itemize}
	\item[] $H$ {\'e} a constante de in{\'e}rcia do gerador em $pu$, na base do sistema;
	\item[] $f_{0}$ {\'e} a base de frequ{\^e}ncia do sistema em $Hz$ ($hertz$);
	\item[] $\delta$ {\'e} o {\^a}ngulo de deslocamento el{\'e}trico em rela{\c c}{\~a}o ao sistema mec{\^a}nico, ou seja, a diferen{\c c}a espacial entre os {\^a}ngulos dos campos magn{\'e}ticos do estator e do rotor no gerador s{\'i}ncrono, em $rad$;
	\item[] $T_{G}$ {\'e} o torque mec{\^a}nico do gerador em $pu$, na base do sistema;
	\item[] $T_{L}$ {\'e} o torque el{\'e}trico da carga em $pu$, na base do sistema;
	\item[] $T_{a}$ {\'e} o torque de acelera{\c c}{\~a}o da rede, em $pu$, na base do sistema.
\end{itemize}

A velocidade do gerador {\'e} dada por:

\begin{equation}
	\label{eq:velg}
	\frac{d \delta}{dt} + \omega_{0} = 2 \pi \cdot f
\end{equation}

Onde,

\begin{itemize}
	\item[] $\omega_{0}$ {\'e} a velocidade s{\'\i}ncrona;
	\item[] $f$ {\'e} a frequ{\^e}ncia corrente.
\end{itemize}

Diferenciando ambos os lados da Equa{\c c}{\~a}o~\ref{eq:velg} em rela{\c c}{\~a}o a $t$:

\begin{equation}
	\label{eq:veldt}
	\frac{d^{2}\delta}{dt^{2}} = 2 \pi \frac{df}{dt}
\end{equation}

Substituindo (\ref{eq:veldt}) em (\ref{eq:inerc}):

\begin{equation}
	\label{eq:dfdt}
	\frac{df}{dt} = \frac{\left(T_{G}-T_{L}\right)f_{0}}{2H} = \frac{T_{a}f_{0}}{2H}
\end{equation}

Valores positivos de $T_{a}$ representam uma acelera{\c c}{\~a}o na frequ{\^e}ncia da rede, enquanto valores negativos, o oposto. A Equa{\c c}{\~a}o~\ref{eq:dfdt} {\'e} utilizada para varia{\c c}{\~o}es na gera{\c c}{\~a}o com carga constante.

O modelo apresentado na Equa{\c c}{\~a}o~\ref{eq:inerc} n{\~a}o inclui fatores de atrito. Para tanto, \citeauthoronline{kundur1994power} \cite{kundur1994power} apresentam uma atualiza{\c c}{\~a}o deste, conforme Equa{\c c}{\~a}o~\ref{eq:damper}, em que adiciona-se um fator proporcional {\`a} velocidade angular.

\begin{equation}
	\label{eq:damper}
	{2H}\frac{d^{2}\delta}{dt^{2}} = T_{a} - D_{L}\frac{d\omega}{dt}
\end{equation}

Onde,

\begin{itemize}
	\item[] $D_{L}$ {\'e} um fator de amortecimento, uma fun{\c c}{\~a}o da composi{\c c}{\~a}o das cargas do sistema.
\end{itemize}

J{\'a} a varia{\c c}{\~a}o na pot{\^e}ncia da carga em fun{\c c}{\~a}o da mudan{\c c}a de frequ{\^e}ncia {\'e}, de acordo com \citeauthor{get6449}, dada por:

\begin{equation}
	\label{eq:plkfdl}
	P_{L} = k\cdot f^{D_{L}}
\end{equation}

Onde,

\begin{itemize}
	\item[] $P_{L}$ {\'e} a pot{\^e}ncia da carga em $pu$;
	\item[] $k$ {\'e} constante;
	\item[] $f$ {\'e} a frequ{\^e}ncia;
	\item[] $D_{L}$ {\'e} o fator de amortecimento, uma fun{\c c}{\~a}o da carga.
\end{itemize}

Como os valores de $f$ est{\~a}o em $pu$, a Equa{\c c}{\~a}o~\ref{eq:ptom} permite reescrever o torque da carga ($T_{L}$) como a pot{\^e}ncia dividida pela frequ{\^e}ncia\footnote{Lembrando que: $\omega_{pu} = \frac{\omega}{\omega_{o}} = \frac{2\pi f}{2\pi f_{o}} = \frac{f}{f_{o}} = f_{pu}$}. Assim:

\begin{equation}
	\label{eq:tlf}
	T_{L} = \frac{P_{L}}{f} = \frac{k \cdot f^{D_{L}}}{f} = k \cdot f^{D_{L}-1}
\end{equation}

Os passos abaixo permitem obter o torque da carga para pequenas mudan{\c c}as de frequ{\^e}ncia (em \textit{pu}):

\[
T_{L} = k \cdot f^{D_{L}-1}
\]

\[
\frac{dT_{L}}{df} = \left(D_{L}-1\right)k \cdot f^{D_{L}-2}
\]

\[
\Delta T_{L} = \left(D_{L}-1\right) k \cdot f^{D_{L}-2} \Delta f
\]

\[
T_{L} + \Delta T_{L} = k \cdot f^{D_{L}-1} + \left(D_{L}-1\right) k \cdot f^{D_{L}-2} \Delta f
\]

Fazendo-se $k \cdot f^{D_{L}-1} = T_{L_{o}}$:

\begin{equation}
	\label{eq:tl}
	T_{L} = T_{L_{o}} \left[1 + \left(D_{L}-1\right)f'\right]
\end{equation}

Onde,

\begin{itemize}
	\item[] $f' = \frac{\Delta f}{f}$ {\'e} a mudan{\c c}a na frequ{\^e}ncia em $pu$;
	\item[] $T_{L_{o}}$ {\'e} o torque inicial em $pu$.
\end{itemize}

Para o gerador, ainda de acordo com \citeauthor{get6449}, obt{\'e}m-se a seguinte equa{\c c}{\~a}o de torque:

\begin{equation}
	\label{eq:tgen}
	T_{G} = \frac{k}{f} = k \cdot f^{-1}
\end{equation}

Analogamente, para pequenas varia{\c c}{\~o}es de frequ{\^e}ncia:

\begin{equation}
	\label{eq:tg}
	T_{G} = T_{G_{o}} \left(1-f'\right)
\end{equation}

Onde,

\begin{itemize}
	\item[] $f'$ {\'e} a varia{\c c}{\~a}o de frequ{\^e}ncia em $pu$;
	\item[] $T_{G_{o}}$ {\'e} o torque inicial do gerador em $pu$;
	\item[] $T_{G}$ {\'e} o torque do gerador em $pu$ ap{\'o}s a mudan{\c c}a.
\end{itemize}

Substituindo (\ref{eq:tg}) e (\ref{eq:tl}) em (\ref{eq:dfdt}) e fazendo-se $D_{T} = T_{G_{o}} + \left(D_{L}-1\right)T_{L_{o}}$:

\begin{equation}
	\label{eq:difeq}
	2H\frac{df'}{dt} + D_{T}f' = T_{a}
\end{equation}

A solu{\c c}{\~a}o desta equa{\c c}{\~a}o diferencial {\'e}:

\begin{equation}
	\label{eq:soldif}
	\Delta \omega = f' = \frac{T_{a}}{D_{T}} \left(1 - e^{-\frac{D_{T}}{2H}t}\right)
\end{equation}

Onde,

\begin{itemize}
	\item[] $f'$ {\'e} a mudan{\c c}a de frequ{\^e}ncia em $pu$ ($\Delta \omega$);
	\item[] $T_{a}$ {\'e} o torque de acelera{\c c}{\~a}o da rede em $pu$ na base da gera{\c c}{\~a}o remanescente;
	\item[] $H$ {\'e} a constante de in{\'e}rcia em $pu$ da gera{\c c}{\~a}o remanescente na base desta.
\end{itemize}

A interpreta{\c c}{\~a}o da Equa{\c c}{\~a}o~\ref{eq:soldif}, que foi desenvolvida e aplicada por \citeauthor{get6449}, implica em uma maneira de calcular numericamente o efeito imediato sobre a frequ{\^e}ncia da rede quando se perde parte da gera{\c c}{\~a}o. Ao mesmo tempo, ela tamb{\'e}m fornece um m{\'e}todo de avaliar o efeito da resposta ao reduzir carga. V{\^e}-se, na Se{\c c}{\~a}o~\ref{ssec:ufls}, como esse resultado pode ser aplicado na constru{\c c}{\~a}o de modelos para rejei{\c c}{\~a}o autom{\'a}tica de carga.

\begin{figure}[!h]
	\centering
	\begin{Large}
		\begin{tikzpicture}
			\sbEntree{E}
			\sbNomLien[0.8]{E}{$T_{G}$}
			\sbCompSum*{a}{E}{-}{-}{+}{}
			\sbDecaleNoeudy[-3]{a}{L}
			\sbNomLien{L}{$T_{L}$}
			\sbBlocL{b}{$\frac{1}{2Hs}$}{a}
			\sbSortie[5]{S}{b}
			\sbRelier{E}{a}
			\sbRelier{b}{S}
			\sbRelier{L}{a}
			\sbNomLien[0.8]{S}{$\Delta \omega$}
			\sbDecaleNoeudy{b}{c}
			\sbBlocr[-1.5]{d}{$D_{T}$}{c}
			\sbRelieryx{b-S}{d}
			\sbRelierxy{d}{a}
		\end{tikzpicture}
	\end{Large}
	\caption[Diagrama de Blocos da Malha de Frequ{\^e}ncia]{Diagrama de Blocos da Malha de Frequ{\^e}ncia [Fonte: Dedu{\c c}{\~a}o]}
	\label{fig:diag1}
\end{figure}

A malha de controle que apresenta o modelo da Equa{\c c}{\~a}o~\ref{eq:difeq} est{\'a} representada nas Figuras~\ref{fig:diag1} e \ref{fig:diag2}. A Tabela~\ref{tab:inercia} apresenta valores t{\'i}picos para a constante de in{\'e}rcia $H$ em $pu$ na base pr{\'o}pria para diversos tipos de geradores.

\begin{figure}[!h]
	\centering
	\begin{Large}
		\begin{tikzpicture}
			\sbEntree{E}
			\sbNomLien[0.8]{E}{$T_{A}$}
			\sbBlocL[5]{a}{$\frac{1}{D_{T}+2Hs}$}{E}
			\sbBlocL[5]{b}{$\frac{\omega_{0}}{s}$}{a}
			\sbSortie[5]{S}{b}
			\sbRelier[$\Delta \omega$]{a}{b}
			\sbRelier{b}{S}
			\sbNomLien[0.8]{S}{$\delta$}
		\end{tikzpicture}
	\end{Large}
	\caption[Diagrama de Blocos da Malha de Frequ{\^e}ncia Simplificado]{Diagrama de Blocos da Malha de Frequ{\^e}ncia Simplificado [Fonte: \citeauthor{kundur1994power}, adaptado]}
	\label{fig:diag2}
\end{figure}

Complementando o desenvolvimento apresentado anteriormente, bem como os valores da Tabela~\ref{tab:inercia}, \citeauthor{amelirole} apresentam um estudo da influ{\^e}ncia de cada um dos par{\^a}metros sobre as equa{\c c}{\~o}es de frequ{\^e}ncia. Esses valores ser{\~a}o utilizados no presente estudo para simular o comportamento din{\^a}mico do sistema proposto.

\begin{table}[!h]
	\begin{center}
		\caption[Valores T{\'i}picos de Constante de In{\'e}rcia para Geradores T{\'e}rmicos e Hidr{\'a}ulicos]{Valores T{\'i}picos de Constante de In{\'e}rcia para Geradores T{\'e}rmicos e Hidr{\'a}ulicos [Fonte: \citeauthor{kundur1994power}, em tradu{\c c}{\~a}o livre]}
		\label{tab:inercia}
	    \vspace{5pt}
		\begin{tabular}{c c}
			\hline
			\textbf{\textbf{Tipo de Unidade Geradora}} & \textbf{$H$}\\
			\hline\hline
			Unidade T{\'e}rmica & \\
			(a) 3.600 rpm (2 p{\'o}los) & $2,5$ a $6,0$ \\
			(b) 1.800 rpm (4 p{\'o}los) & $4,0$ a $10,0$ \\
			\hline\hline
			Unidade Hidr{\'a}ulica & $2,0$ a $4,0$ \\
			\hline
		\end{tabular}
	\end{center}
\end{table}

\section{Solu{\c c}{\~o}es Existentes} \label{sec:exist}

\subsection{\textbf{UFLS} \--- Sistemas Baseados em Sub-Frequ{\^e}ncia} \label{ssec:ufls}

O valor de $\Delta \omega$ fornecido na Equa{\c c}{\~a}o~\ref{eq:soldif} juntamente aos valores de refer{\^e}ncia fornecidos na Tabela~\ref{tab:undf}, permite a constru{\c c}{\~a}o de dispositivos de rejei{\c c}{\~a}o de carga baseados em sub-frequ{\^e}ncia, tipo \textbf{UFLS} (\textit{Under Frequency Load Shedding}). Esses dispositivos t{\^e}m em seu favor a vantagem de disporem do mesmo valor de refer{\^e}ncia medido em quaisquer pontos da rede, sendo especialmente vantajosos para redes de grande porte e, principalmente, com gera{\c c}{\~a}o distribu{\'\i}da \cite{shekari2018}. A varia{\c c}{\~a}o na frequ{\^e}ncia , expressa por $\Delta \omega$, serve de gatilho para uma primeira etapa de desligamentos e, em seguida, {\'e} realizada uma reavalia{\c c}{\~a}o da tend{\^e}ncia de comportamento da frequ{\^e}ncia, repetindo at{\'e} que esta comece a recuperar-se. \citeauthor{get6449} recomenda configurar no m{\'\i}nimo tr{\^e}s e no m{\'a}ximo cinco etapas de desligamento, atrav{\'e}s de um processo iterativo.

O diagrama de blocos na Figura~\ref{fig:diag3} apresenta um sistema de controle \textbf{UFLS} baseado na Equa{\c c}{\~a}o~\ref{eq:soldif} para plantas termel{\'e}tricas a vapor. Este modelo \footnote{Este modelo foi proposto originalmente por \citeauthor{65898} e serve apenas para ilustrar, pois considera um cen{\'a}rio de gera{\c c}{\~a}o com predomin{\^a}ncia de turbinas a vapor, que n{\~a}o {\'e} o caso das redes estudadas aqui, que t{\^e}m predomin{\^a}ncia de turbinas a g{\'a}s e motores a diesel.} acrescenta a rea{\c c}{\~a}o da acelera{\c c}{\~a}o das turbinas na malha de controle e demonstra uma forma de incluir os efeitos da rejei{\c c}{\~a}o de cargas diretamente na malha.

\begin{figure}[!h]
	\centering
	\begin{Large}
		\begin{tikzpicture}
		\sbEntree{E}
		\sbNomLien[0.8]{E}{$P_{a}$}
		\sbComp*{a}{E}
		\sbBlocL{b}{$\frac{1}{D+2Hs}$}{a}
		\sbSortie[5]{S}{b}
		\sbRelier{E}{a}
		\sbRelier{b}{S}
		\sbNomLien[0.8]{S}{$\Delta \omega$}
		\sbDecaleNoeudy{b}{c}
		\sbBlocr[-3]{d}{$\frac{K_{M}\left(1+F_{H}T_{R}s\right)}{R\left(1+T_{R}s\right)}$}{c}
		\sbRelieryx{b-S}{d}
		\sbRelierxy[$P_{m}$]{d}{a}
		\end{tikzpicture}
	\end{Large}
	\caption[\textbf{UFLS}]{\textbf{UFLS} [Fonte: \apud{65898}{amelirole}]}
	\label{fig:diag3}
\end{figure}

Onde,

\begin{itemize}
	\item[] $K_{M}$ {\'e} o ganho da malha de controle da frequ{\^e}ncia;
	\item[] $F_{H}$ {\'e} a pot{\^e}ncia das turbinas;
	\item[] $T_{R}$ {\'e} a constante de tempo de reaquecimento;
	\item[] $P_{m}$ {\'e} a pot{\^e}ncia mec{\^a}nica da turbina;
	\item[] $P_{a}$ {\'e} a pot{\^e}ncia de acelera{\c c}{\~a}o.
\end{itemize}

\subsection{\textbf{UVLS} \--- Sistemas Baseados em Sub-Tens{\~a}o} \label{ssec:uvls}

A alternativa cl{\'a}ssica aos modelos baseados na sub-freq{\^e}ncia, \textbf{UFLS}, {\'e} a utiliza{\c c}{\~a}o de modelos \textbf{UVLS} (\textit{Under Voltage Load Shedding}), baseados na sub-tens{\~a}o \cite{laghari2014}. Esses {\'u}ltimos apresentam algumas desvantagens como, por exemplo, a facilidade de controlar a tens{\~a}o atrav{\'e}s dos \textit{taps} de transformadores que dificulta a avalia{\c c}{\~a}o ou, ainda, o fato da rede poder apresentar valores de tens{\~a}o variados em pontos distintos. Estes modelos n{\~a}o ser{\~a}o detalhados aqui, entretanto, cabe mencionar que h{\'a} modelos h{\'\i}bridos para trabalhar tanto com sub-tens{\~a}o quanto sub-frequ{\^e}ncia em sistemas distribu{\'\i}dos \cite{ye2015, qing2016}.

\citeauthor{yu2016liu} propuseram um modelo preditivo baseado em \textbf{UVLS}, onde, utilizando a matriz jacobiana \cite{grainger2016power,monticelli1983fluxo} do fluxo de pot{\^e}ncia da rede, buscam valores pr{\'o}ximos de um valor m{\'\i}nimo para definir a barra como potencialmente em risco e, assim, escolher onde disparar o algoritmo de rejei{\c c}{\~a}o de carga.

Os modelos cl{\'a}ssicos apresentados aqui ajudam a resolver o problema de estabilidade el{\'e}trica frente ao cen{\'a}rio de conting{\^e}ncia de perda de gera{\c c}{\~a}o, estimando a quantidade de carga a ser desligada, mas n{\~a}o resolvem outra quest{\~a}o: quais cargas devem ser desligadas? \'{E} neste ponto que as solu{\c c}{\~o}es come{\c c}am a divergir.

A solu{\c c}{\~a}o cl{\'a}ssica adotada na ind{\'u}stria, a exemplo das instala{\c c}{\~o}es das plantas de facilidades el{\'e}tricas nas plataformas de petr{\'o}leo constru{\'\i}das at{\'e} os anos 2000 (ao menos no Brasil), consiste em uma tabela fixa, determinada pelo engenheiro respons{\'a}vel do sistema, ficando a disposi{\c c}{\~a}o dos operadores para consulta (e modifica{\c c}{\~a}o, mediante autoriza{\c c}{\~a}o pr{\'e}via). Embora esse modelo ofere{\c c}a um excelente ganho de velocidade, est{\'a} longe de ser otimizado, pois ignora particularidades e, principalmente, mudan{\c c}as din{\^a}micas que surgem em tempo real na opera{\c c}{\~a}o. Esta limita{\c c}{\~a}o abriu espa{\c c}o para o desenvolvimento dos modelos \textbf{ILS} (\textit{Intelligent Load Shedding}), nome dado aos Sistemas Inteligentes de Rejei{\c c}{\~a}o de Cargas, como ilustrado na Figuar~\ref{fig:ilsdiag}.

\begin{figure}[!h]
	\centering
	\includegraphics[width=\linewidth]{figuras/ilsdiag}
	\caption[\textbf{ILS}]{\textbf{ILS} [Fonte: \citeauthor{1518342}, adaptado]}
	\label{fig:ilsdiag}
\end{figure}

\subsection{\textbf{ILS} \--- Sistemas Inteligentes} \label{ssec:ils}

H{\'a} diversos tipos de modelos \textbf{ILS}; alguns determin{\'\i}sticos, que utilizam ranqueamento \cite{wang2014, verayiah2015}; outros baseados em algoritmos de minimiza{\c c}{\~a}o de custos (ou maximiza{\c c}{\~a}o de benef{\'\i}cios), como algoritmos gen{\'e}ticos; h{\'a} os que se baseiam em L{\'o}gica \textit{Fuzzy}; e, ainda, as \textbf{ANN} (\textit{Artificial Neural Network}), que s{\~a}o as Redes Neurais Artificiais \cite{yan2017, santos2014}. O objetivo destes modelos \cite{7046661} {\'e} agregar a experi{\^e}ncia do engenheiro que projeta o sistema, de forma a otimizar a decis{\~a}o tomada. Assim, o que a maioria dos modelos \textbf{ILS} t{\^e}m em comum {\'e} a utiliza{\c c}{\~a}o de {\'\i}ndices de ranqueamento, pesos, ou, em ingl{\^e}s, \textit{weights}. Esses {\'\i}ndices s{\~a}o definidos pelo projetista do sistema em lugar da tabela est{\'a}tica, geralmente dentro de uma escala de valores definida, sendo, por exemplo, n{\'u}meros inteiros entre $0$ e $10$.

Em \cite{B574}, encontra-se um modelo determin{\'\i}stico que utiliza fatores combinados e classifica{\c c}{\~a}o direta. Assim, em vez de definir pesos atribu{\'\i}dos {\`a}s cargas definindo a import{\^a}ncia geral, eram atribu{\'\i}dos pesos por cen{\'a}rios e esses ficavam dispon{\'\i}veis ao operador para serem escolhidos durante a opera{\c c}{\~a}o, gerando a flexibilidade de escolher a prioridade por sistema dentro da planta industrial. Em uma aplica{\c c}{\~a}o mais generalizada, poderia ser definido em termos de {\'a}rea f{\'\i}sica de distrubui{\c c}{\~a}o em vez de sistema.  Estes {\'\i}ndices s{\~a}o combinados a outros m{\'e}todos dentro do algoritmo adotado para gerar uma classifica{\c c}{\~a}o final, que pode variar de acordo com o cen{\'a}rio ou, ainda, a subesta{\c c}{\~a}o onde o \textbf{ILS} ser{\'a} disparado. Esta classifica{\c c}{\~a}o era aplicada a um modelo matem{\'a}tico que combinava outros fatores, sendo o principal, a pot{\^e}ncia el{\'e}trica medida em tempo real. Entretanto, as limita{\c c}{\~o}es que ele possui e por conta de avan{\c c}os nos recursos computacionais dispon{\'\i}veis, houve a reformula{\c c}{\~a}o completa desta filosofia.

\section{Implementa{\c c}{\~o}es} \label{sec:casos}

O Governo brasileiro, atrav{\'e}s da \citeauthoronline{aneel2015}, estabeleceu o \textbf{ERAC} (Esquema Regional de Al{\'\i}vio de Carga por Sub-Frequ{\^e}ncia), definindo como:

\begin{quote}
	``Sistema de prote{\c c}{\~a}o que, por meio do desligamento autom{\'a}tico e escalonado de blocos de carga, utilizando rel{\'e}s de frequ{\^e}ncia, minimiza os efeitos de subfrequ{\^e}ncia decorrentes de perda de grandes blocos de gera{\c c}{\~a}o.'' \cite[p.~156]{aneel2015}
\end{quote}

A partir da defini{\c c}{\~a}o do \textbf{ERAC}, coube ao \citeauthor{iogcbr02} sua regulamenta{\c c}{\~a}o e implementa{\c c}{\~a}o. Este define os ajustes do \textbf{ERAC} por regi{\~a}o ou {\'a}rea el{\'e}trica, fornecendo tabelas com valores para cada uma por faixa de frequ{\^e}ncia, incluindo bancos de capacitores em subesta{\c c}{\~o}es. H{\'a} duas formas definidas para atua{\c c}{\~a}o do \textbf{ERAC}: valor absoluto e queda acentuada na frequ{\^e}ncia. O \citeauthoronline{iogcbr02} \cite[item 2.1.3]{iogcbr02} define ainda o tempo de atua{\c c}{\~a}o para disparo do est{\'a}gio, sendo cinco est{\'a}gios para cada tabela de atua{\c c}{\~a}o. Este tempo est{\'a} definido como nove ciclos el{\'e}tricos ($150ms$), sendo tr{\^e}s para detec{\c c}{\~a}o e seis para abertura dos disjuntores. Assim, dependendo da velocidade da queda de frequ{\^e}ncia, do valor que esta atinja e da {\'a}rea de opera{\c c}{\~a}o onde ocorra, uma etapa de rejei{\c c}{\~a}o de cargas {\'e} disparada para o percentual de carga correspondente nas tabelas encontradas em \cite{iogcbr02}.

Apesar do modelo consolidado no Brasil, diversos estudos ainda est{\~a}o em curso buscando novas metodologias. Por exemplo, \citeauthor{li2006} estudaram a influ{\^e}ncia das caracter{\'\i}sticas das cargas, como motores s{\'\i}ncronos ou imped{\^a}ncia fixa, e apresentaram seus efeitos sobre a frequ{\^e}ncia em cen{\'a}rios de sub-tens{\~a}o, utilizando simula{\c c}{\~a}o para realizar um estudo de caso da rede el{\'e}trica da cidade de Beijing, no norte da China.

J{\'a} \citeauthor{kucuk2018} modela um sistema industrial, realizando estudo de caso considerando uma refinaria de petr{\'o}leo. \citeauthoronline{kucuk2018} estuda os efeitos do aumento na demanda dos geradores visando aliviar a carga para prevenir uma poss{\'\i}vel sobrecarga. Este caso {\'e} interessante, pois as refinarias t{\^e}m muito em comum com plataformas de produ{\c c}{\~a}o, embora a grande diferen{\c c}a seja a conex{\~a}o ao sistema de transmiss{\~a}o que as refinarias geralmente disp{\~o}em. Entretanto, as caracter{\'\i}sticas das plantas industriais, em oposi{\c c}{\~a}o {\`a}s instala{\c c}{\~o}es comerciais ou sistemas urbanos de distribui{\c c}{\~a}o, t{\^e}m como cargas majorit{\'a}rias motores de grande porte. Assim, \citeauthor{ye2015zhe} comparam o efeito de descartar motores em compara{\c c}{\~a}o {\`a} mesma quantidade de outros tipos de carga, demonstrando que descartar motores primeiro oferece uma vantagem maior na recupera{\c c}{\~a}o, tanto da tens{\~a}o quanto da frequ{\^e}ncia da rede.

Al{\'e}m dos efeitos el{\'e}tricos, h{\'a} ainda um efeito econ{\^o}mico no desligamento. Neste trabalho, como o foco est{\'a} nos sistemas isolados, a quest{\~a}o econ{\^o}mica resume-se ao custo de parada dos equipamentos e da sua produ{\c c}{\~a}o relativa. Entretanto, em sistemas de transmiss{\~a}o, distribui{\c c}{\~a}o e comercializa{\c c}{\~a}o de energia, h{\'a} o valor da pr{\'o}pria energia n{\~a}o entregue. Assim, \citeauthor{tikdari2015} realizaram um estudo onde contabilizaram, junto a outras restri{\c c}{\~o}es, o custo marginal da energia. Tal linha de racioc{\'\i}nio tamb{\'e}m foi seguida por \citeauthor{paul2017}.

Outro aspecto a ser considerado nos sistemas \textbf{ILS} {\'e} a velocidade em que as solu{\c c}{\~o}es s{\~a}o computadas. Em outras palavras, h{\'a} a necessidade de adequar os algoritmos e limita{\c c}{\~o}es de hardware existentes {\`a} velocidade requerida para a atua{\c c}{\~a}o. Ali{\'a}s, este tem sido o real motivo para as redes de grande porte utilizarem os rel{\'e}s de atua{\c c}{\~a}o com tabelas est{\'a}ticas. Os modelos puramente classificadores, como o que foi proposto em \cite{B574}, trazem consigo a vantagem de ter seus c{\'a}lculos computados offline, de forma que, na necessidade de atua{\c c}{\~a}o, os resultados j{\'a} estejam dispon{\'\i}veis. Face a esta limita{\c c}{\~a}o, \citeauthor{wester2014} propuseram um esquema de atua{\c c}{\~a}o r{\'a}pida baseado em medi{\c c}{\~o}es de campo, avaliando o balan{\c c}o entre gera{\c c}{\~a}o e carga em tempo real.

\chapter{Metodologia} \label{cap:metod}

Visando determinar um crit{\'e}rio para sele{\c c}{\~a}o das cargas a serem descartadas que atenda aos requisitos de velocidade (estando dispon{\'\i}vel sempre que houver necessidade de atua{\c c}{\~a}o do rel{\'e}), praticidade (visando otimizar a escolha das cargas para preservar as mais importantes) na forma din{\^a}mica e adaptativa (buscando retirar o menor n{\'u}mero poss{\'\i}vel de cargas, minimizando os impactos sobre a rede), o modelo adotado neste trabalho baseia-se em meta-heur{\'\i}stica, tendo como filosofia a atua{\c c}{\~a}o preditiva.

A atua{\c c}{\~a}o do sistema de rejei{\c c}{\~a}o seguir{\'a} a metodologia \textbf{UFLS}, conforme apresentada por \citeauthor{amelirole}, sendo a mesma adotada no Sistema El{\'e}trico Brasileiro \cite{aneel2015}. Um ajuste em cinco etapas ser{\'a} utilizado, podendo se utilizar menos etapas conforme a implementa{\c c}{\~a}o. Assim, a tabela de rejei{\c c}{\~a}o conter{\'a} cinco blocos de cargas, sendo um para cada etapa, para atua{\c c}{\~a}o em cascata, conforme haja necessidade.

A partir da defini{\c c}{\~a}o precedente, dada a etapa de desligamento, a entrada do algoritmo que efetuar{\'a} a ordena{\c c}{\~a}o de cargas para gerar sua respectiva tabela receber{\'a} como entrada o ajuste de pot{\^e}ncia de cada etapa e as cargas em opera{\c c}{\~a}o com suas respectivas demandas e pesos de opera{\c c}{\~a}o. Como valores de ajuste (\textit{set points}), este contar{\'a} com os crit{\'e}rios de opera{\c c}{\~a}o e seus respectivos pesos globais e individuais, conforme ser{\'a} detalhado mais adiante na Sube{\c c}{\~a}o~\ref{subsec:f3}. Utilizando esta entrada e estes ajustes, um algoritmo de busca meta-heur{\'\i}stica far{\'a} sucessivas permuta{\c c}{\~o}es visando otimizar a tabela. Ap{\'o}s um intervalo de tempo definido, o sistema atualizar{\'a} as leituras de campo, encerrando a busca para os par{\^a}metros atuais e iniciando uma nova a partir dos valores recebidos.

O crit{\'e}rio de otimiza{\c c}{\~a}o {\'e} uma fun{\c c}{\~a}o l{\'o}gica que constitui a proposta deste trabalho, a qual chamaremos de Fun{\c c}{\~a}o Objetivo\footnote{Ou Fun{\c c}{\~a}o de Custo}, estando na Se{\c c}{\~a}o~\ref{sec:obj}.

\section{Meta-Heur{\'\i}stica de Busca} \label{sec:meth}

A busca que ser{\'a} realizada aqui assemelha-se ao problema do Caixeiro Viajante \cite{applegate2006traveling} que consiste em, dada uma lista de cidades e uma matriz com os custos de deslocamento entre elas, encontrar a ordem em que um caixeiro viajante consegue percorrer todas as cidades sem repeti{\c c}{\~a}o, ao menor custo. No problema original, o {\'u}ltimo ponto {\'e} igual ao primeiro, pois o caixeiro retorna ao ponto de origem. J{\'a} no problema das cargas, n{\~a}o h{\'a} nenhuma repeti{\c c}{\~a}o. Portanto, quaisquer m{\'e}todos de busca que possam resolver este problema cl{\'a}ssico podem ser adotados aqui, sendo a escolha influenciada mais pelo tempo de converg{\^e}ncia para um {\'o}timo local do que para uma solu{\c c}{\~a}o exaustiva. Neste trabalho ser{\'a} adotado o m{\'e}todo \textbf{VND} \--- \textit{Variable Neighborhood Descent} \cite{hansen2001449,hansen2019}, por oferecer r{\'a}pida converg{\^e}ncia para aplica{\c c}{\~a}o em tempo real, caracterizado pela Equa{\c c}{\~a}o~\ref{eq:VNDobj}, cuja solu{\c c}{\~a}o alcan{\c c}a-se atrav{\'e}s do Algoritmo~\ref{alg:vnd}.

\begin{equation} \label{eq:VNDobj}
    min \left\{ f \left( x \right) | x \in \mathcal{X}, \mathcal{X} \subseteq \mathcal{S} \right\}
\end{equation}

Onde,

\begin{itemize}
    \item[] $x$ {\'e} uma solu{\c c}{\~a}o buscada (avaliada);
    \item[] $f \left( x \right)$ {\'e} a fun{\c c}{\~a}o de custo ou fun{\c c}{\~a}o objetivo;
    \item[] $\mathcal{X}$ {\'e} o espa{\c c}o de busca (solu{\c c}{\~o}es avali{\'a}veis);
    \item[] $\mathcal{S}$ {\'e} o espa{\c c}o total de solu{\c c}{\~o}es poss{\'\i}veis.
\end{itemize}

\begin{algorithm}[!h]
	\caption[\textbf{VND} \--- \textit{Variable Neighborhood Descent}]{\textbf{VND} \--- \textit{Variable Neighborhood Descent} [Fonte: \citeauthor{hansen2019}, Figura~6.1, adaptado]}
	\label{alg:vnd}
	\begin{algorithmic}
		\STATE{In{\'\i}cio}
		\STATE{Seleciona o conjunto de estruturas de vizinhan{\c c}a $\mathcal{N}_{k}$, para $k=1, \dots, k_{max}$, que ser{\'a} utilizado na busca}
		\STATE{Encontra uma solu{\c c}{\~a}o inicial $x$}
		\STATE{Escolhe uma condi{\c c}{\~a}o de parada}
		\WHILE{Condi{\c c}{\~a}o de parada n{\~a}o satisfeita}
			\STATE{$k \leftarrow 1$}
			\WHILE{$k \ne k_{max}$}
				\STATE{Gera um ponto $x'$ aleatoriamente a partir da $k-esima$ vizinhan{\c c}a de $x$ ($x' \in \mathcal{N}_{k}\left( x \right)$)}
				\IF{$f \left( x' \right) < f \left( x \right)$}
				    \STATE{$x \leftarrow x'$}
				    \STATE{$k \leftarrow 1$}
				\ELSE
				    \STATE{$k \leftarrow k + 1$}
				\ENDIF
			\ENDWHILE
		\ENDWHILE
	\end{algorithmic}
\end{algorithm}

Para utilizar o \textbf{VND}, portanto, deve-se escolher estruturas de vizinhan{\c c}a de ordens sucessivas e alternar la{\c c}os de busca com a menor vizinhan{\c c}a no la{\c c}o mais interno. Assim, o primeiro passo {\'e} definir as estruturas de vizinhan{\c c}a.

A vizinhan{\c c}a de primeira ordem, $\mathcal{N}_{1}$, ser{\'a} definida como a troca entre elementos com {\'\i}ndices $i$ e $j$ limitados a $n$, ou seja, uma altera{\c c}{\~a}o na ordem de duas cargas destinadas ao desligamento. Isso permite ao termo da fun{\c c}{\~a}o de custo que ir{\'a} tratar da prioridade na ordena{\c c}{\~a}o encontrar uma ordem mais adequada para um conjunto de $n$ cargas, pois se a frequ{\^e}ncia equilibrar-se antes da {\'u}ltima carga no conjunto ser descartada, conv{\'e}m que a carga preservada seja a mais importante. Trata-se de uma otimiza{\c c}{\~a}o do conjunto j{\'a} encontrado. O melhor resultado obtido ser{\'a} considerado um m{\'\i}nimo local.

A vizinhan{\c c}a de segunda ordem, $\mathcal{N}_{2}$, ser{\'a} definida como a troca entre uma carga de {\'\i}ndice $i$ limitado a $n$ e {\'\i}ndice $j$ superior a $n$. Na pr{\'a}tica, trata-se de uma troca entre um elemento que seria descartado por outro que se manteria. Na pr{\'a}tica, isso significa migrar de uma vizinhan{\c c}a para outra em busca de um novo m{\'\i}nimo local.

Uma vizinhan{\c c}a de terceira ordem, $\mathcal{N}_{3}$, seria a troca entre dois elementos de {\'\i}ndices maiores que $n$. Entretanto, trocar a ordem de duas cargas fora do conjunto a ser desligado n{\~a}o faz sentido e, portanto, n{\~a}o ser{\'a} definido.

Uma alternativa ao \textbf{VND} {\'e} o \textit{Iterated Local Search} \cite{lourencco2019}. A implementa{\c c}{\~a}o deste m{\'e}todo requer uma perturba{\c c}{\~a}o quando a solu{\c c}{\~a}o n{\~a}o apresenta mais melhoria com a varia{\c c}{\~a}o das vizinhan{\c c}as locais. O objetivo desta perturba{\c c}{\~a}o {\'e} fugir de valores m{\'\i}nimos locais, procurando uma regi{\~a}o mais distante, fora das estruturas de vizinhan{\c c}a, que pode conter um m{\'\i}nimo local mais vantajoso. Esta perturba{\c c}{\~a}o ser{\'a} definida como uma invers{\~a}o total na ordem das cargas na melhor solu{\c c}{\~a}o encontrada\footnote{Embora esta perturba{\c c}{\~a}o seja parte do algoritmo \textbf{ILS}, ela n{\~a}o foi implementada neste trabalho, sendo utilizado, portanto, o \textbf{VND} apenas. Ela est{\'a} apresentada aqui como uma implementa{\c c}{\~a}o alternativa para melhoria em trabalhos futuros.}.

Sempre que uma solu{\c c}{\~a}o resultar em um custo inferior ao melhor custo obtido, este ser{\'a} atualizado e a busca ser{\'a} reiniciada utilizando este novo resultado como solu{\c c}{\~a}o inicial, ou seja, a busca retorna ao espa{\c c}o da vizinhan{\c c}a $\mathcal{N}_{1}$, conforme o Algoritmo~\ref{alg:vnd}.

Como existe uma in{\'e}rcia no sistema em opera{\c c}{\~a}o normal, a cada nova leitura, no caso de n{\~a}o haver mudan{\c c}a topol{\'o}gica ou entrada e sa{\'\i}da de elementos da rede, a solu{\c c}{\~a}o da busca anterior ser{\'a} utilizada como ponto de partida (solu{\c c}{\~a}o inicial) para a busca seguinte. Entretanto, cada vez que uma carga entra ou sai de opera{\c c}{\~a}o, esta mudan{\c c}a altera a topologia da rede. Assim, havendo mudan{\c c}a na topologia, o vetor de solu{\c c}{\~o}oes que cont{\'e}m as cargas ativas, constituindo o espa{\c c}o de busca, altera seu comprimento, fazendo com que o Algoritmo~\ref{alg:vnd} necessite ser reiniciado. Esta caracter{\'\i}stica aparece nos resultados, j{\'a} que, em cada novo in{\'\i}cio, o resultado converge de forma mais r{\'a}pida, com uma tend{\^e}ncia a se estabilizar em torno de uma solu{\c c}{\~a}o. A Se{\c c}{\~a}o~\ref{sec:anares} apresentar{\'a} em detalhes como isso ocorre na pr{\'a}tica.

A forma final do esquema adotado neste trabalho est{\'a} apresentada no Algoritmo~\ref{alg:lsgeral}, que apresenta as etapas gerais da metodologia proposta, e a busca com a meta-heur{\'\i}stica adotada, no Algoritmo~\ref{alg:lsbusca}.



\begin{algorithm}[h]
	\caption{Algoritmo de Busca para Rejei{\c c}{\~a}o de Cargas - Vis{\~a}o Geral}
	\label{alg:lsgeral}
	\begin{algorithmic}
		\STATE{In{\'\i}cio}
		\STATE{Ordena as cargas em opera{\c c}{\~a}o utilizando os fatores de opera{\c c}{\~a}o como solu{\c c}{\~a}o inicial (classifica{\c c}{\~a}o direta)}
		\WHILE{Sistema Operando}
			\STATE{Recebe leitura do campo com o estado do sistema}
			\WHILE{N{\~a}o h{\'a} nova leitura dispon{\'\i}vel}
				\STATE{Aciona algoritmo de busca}
			\ENDWHILE
		\ENDWHILE
	\end{algorithmic}
\end{algorithm}

\begin{algorithm}[h]
	\caption{Algoritmo de Busca para Rejei{\c c}{\~a}o de Cargas - Meta-Heur{\'\i}stica}
	\label{alg:lsbusca}
	\begin{algorithmic}
		\STATE{In{\'\i}cio}
		\STATE{Recebe estado do sistema}
		\STATE{Salva a solu{\c c}{\~a}o atual como melhor}
		\STATE{Calcula e armazena custo da solu{\c c}{\~a}o atual como melhor}
		\WHILE{Estado se mant{\'e}m}
			\WHILE{Melhor solu{\c c}{\~a}o se altera}
				\FOR{$i$ dentro do n{\'u}mero de cargas ativas}
					\FOR{j dentro do n{\'u}mero de cargas ativas e $j>i$}
						\STATE{$S$ $\leftarrow$ Troca a ordem das cargas $i$ e $j$}
						\IF{Custo de $S$ menor que melhor custo}
							\STATE{Atualiza a melhor solu{\c c}{\~a}o}
							\STATE{Atualiza melhor custo}
							\STATE{Atualiza os rel{\'e}s do sistema de atua{\c c}{\~a}o}
						\ENDIF
					\ENDFOR
				\ENDFOR
			\ENDWHILE
		\ENDWHILE
	\end{algorithmic}
\end{algorithm}

\section{Fun{\c c}{\~a}o Objetivo} \label{sec:obj}

Como mencionado anteriormente, esta fun{\c c}{\~a}o representa efetivamente a contribui{\c c}{\~a}o deste trabalho no sentido de propor uma solu{\c c}{\~a}o para o problema de Rejei{\c c}{\~a}o de Cargas. Diversos trabalhos, conforme indicado no Cap{\'\i}tulo~\ref{cap:revbib}, utilizam sistema \textbf{ILS}, muitos dos quais se baseiam em crit{\'e}rios de busca ou ordena{\c c}{\~a}o. O que os diferencia {\'e}, justamente, o crit{\'e}rio utilizado nesta busca. Assim, a Equa{\c c}{\~a}o~\ref{eq:objsimp} ser{\'a} o crit{\'e}rio adotado aqui.

\begin{equation} \label{eq:objsimp}
	\mathcal{F}\left( x, P \right) = \sum_{i=1}^{4}{C_{i} \times f_{i} \left( x, P \right)}
\end{equation}

Onde,

\begin{itemize}
	\item[] $ x $ {\'e} uma solu{\c c}{\~a}o que consiste em uma lista de valores contendo os {\'\i}ndices das cargas do sistema em uma ordem particular;
	\item[] $ P $ {\'e} a pot{\^e}ncia para descartar no bloco;
	\item[] $ f_{1}\left(x, P \right) $ {\'e} a penalidade por diferen{\c c}a de pot{\^e}ncia retirada;
	\item[] $ f_{2}\left(x, P \right) $ {\'e} a penalidade por n{\'u}mero de cargas retiradas;
	\item[] $ f_{3}\left(x, P \right) $ {\'e} a aplica{\c c}{\~a}o dos crit{\'e}rios de import{\^a}ncia para a opera{\c c}{\~a}o;
	\item[] $ f_{4}\left(x, P \right) $ {\'e} o efeito de ordena{\c c}{\~a}o das cargas retiradas;
	\item[] $ C_{i} $ {\'e} o fator de pondera{\c c}{\~a}o de $ f_{i}\left(x, P \right) $ no somat{\'o}rio.
\end{itemize}

As cinco etapas de desligamento configuradas atuam sucessivamente, mas n{\~a}o necessariamente at{\'e} a {\'u}ltima. Assim, se, por exemplo, a primeira etapa contemplar $10\%$ da carga, e a segunda, $15\%$, ao atuar o primeiro bloco, se a frequ{\^e}ncia come{\c c}ar a recuperar-se, o segundo bloco n{\~a}o chegar{\'a} a atuar, caso contr{\'a}rio, ser{\'a} descartado um montante de $15\%$ aplicado sobre os $90\%$ que restaram na primeira etapa ou seja, $13,5\%$ do montante inicial. Embora cada etapa atue de forma independente, cada bloco depende da defini{\c c}{\~a}o do bloco anterior, j{\'a} que as cargas que comp{\~o}em um bloco n{\~a}o podem ser contempladas no bloco seguinte. Essa depend{\^e}ncia pode ser um problema, pois cada etapa precisa ser definida para come{\c c}ar a busca seguinte. Para contornar esta depend{\^e}ncia, ser{\'a} utilizado um artif{\'\i}cio l{\'o}gico. Dada uma classifica{\c c}{\~a}o de cargas, as primeiras ser{\~a}o alocadas para o primeiro bloco at{\'e} que se tenha os $15\%$ ajustados, as seguintes ser{\~a}o alocadas para o segundo bloco at{\'e} atingir o montante de $13,5\%$ calculado para o ajuste de $15\%$, e assim sucessivamente at{\'e} o quinto bloco. Para equilibrar a classifica{\c c}{\~a}o dos blocos e evitar que o {\'u}ltimo bloco que tem menor probabilidade de atua{\c c}{\~a}o seja avaliado com o mesmo peso do primeiro que dever{\'a} necessariamente atuar, o custo total ser{\'a} uma soma ponderada do custo dos blocos, com fator unit{\'a}rio para o primeiro bloco e descr{\'e}scimo de $0,2$ para cada bloco posterior, conforme consta na Equa{\c c}{\~a}o~\ref{eq:pond}.

\begin{equation} \label{eq:pond}
	f\left( x \right) = \sum_{j=1}^{5}{\sum_{i=1}^{4}{\left(1,2-0,2\times j\right) \times C_{i} \times f_{i} \left( x, P_{j} \right)}}
\end{equation}

\subsection{Propaga{\c c}{\~a}o de Efeitos} \label{subsec:f0}

A fun{\c c}{\~a}o de custo retorna, virtualmente, dois valores, pois uma informa{\c c}{\~a}o obtida na Equa{\c c}{\~a}o~\ref{eq:f0} deve ser levada adiante junto com o custo da solu{\c c}{\~a}o: o valor de $n$ que apresenta o menor custo, ou seja, a quantidade de cargas descartada numa solu{\c c}{\~a}o particular. Esta considera{\c c}{\~a}o torna-se importante, pois causa uma diferen{\c c}a conceitual entre permutar elementos inclusos no conjunto de cargas que ser{\~a}o descartadas e elementos n{\~a}o inclusos. Conforme veremos, isso nos permite definir as vizinhan{\c c}as de forma menos habitual.

\begin{equation} \label{eq:f0}
	f_{0} \left( x, P \right) = \vec{\Pi}
\end{equation}

A fun{\c c}{\~a}o estabelecida na Equa{\c c}{\~a}o~\ref{eq:f0} {\'e} puramente l{\'o}gica e, embora n{\~a}o entre no somat{\'o}rio diretamente, norteia os tr{\^e}s primeiros termos deste. Ela utiliza a matriz de correla{\c c}{\~a}o ou propaga{\c c}{\~a}o. Esta matriz {\'e} bin{\'a}ria, e cada linha corresponde a uma carga indica se o seu desligamento tamb{\'e}m ocasiona o desligamento da carga correspondente {\`a} coluna. Assim, a linha que corresponde a um painel, por exemplo, ter{\'a} todas as colunas correspondentes as suas cargas com valor $True$ (ou seja, $1$). Analogamente, todos os valores na diagonal principal ser{\~a}o verdadeiros.

A sa{\'\i}da desta fun{\c c}{\~a}o ser{\'a} um vetor, tamb{\'e}m bin{\'a}rio (booleano), cujo n{\'u}mero de elementos verdadeiros ser{\'a} o total de cargas desligadas e a posi{\c c}{\~a}o destes corresponde a quais s{\~a}o estas.

Seu c{\'a}lculo consiste unicamente em uma opera{\c c}{\~a}o $ou$ entre os vetores formados pelas $n$ primeiras linhas da matriz de correla{\c c}{\~a}o.

\subsection{Diferen{\c c}a de Pot{\^e}ncia} \label{subsec:f1}

\begin{equation} \label{eq:f1}
	f_{1} \left( x, P \right) = \frac{1}{e^{2}} \frac{\left( P - \left< \vec{P_{x}}, \vec{\Pi} \right> \right)^{2}}{P^{2}}
\end{equation}

Onde,

\begin{itemize}
	\item[] $ P $ {\'e} o valor (escalar) de refer{\^e}ncia para a carga a ser retirada na etapa de desligamento. \'{E} poss{\'\i}vel que seja necess{\'a}rio somar uma unidade no denominador para evitar uma poss{\'\i}vel divis{\~a}o por $0$ caso se trabalhe com inteiros, mas a exclus{\~a}o direta da busca para o caso de uma gera{\c c}{\~a}o nula {\'e} mais eficiente;
	\item[] $ \vec{P_{x}} $ {\'e} o vetor contendo os valores de pot{\^e}ncia das cargas cargas em $x$;
	\item[] $ \vec{\Pi} $ {\'e} o vetor calculado em $f_{0}\left(x, P \right)$;
	\item[] A opera{\c c}{\~a}o $\left< \vec{P_{x}}, \vec{\Pi} \right>$ {\'e} o produto interno entre os vetores.
	\item[] $ e $ {\'e} a toler{\^a}ncia para aplica{\c c}{\~a}o da penalidade. Este valor foi ajustado em $1\%$ nas simula{\c c}{\~o}es.
\end{itemize}

Para realizar este c{\'a}lculo, {\'e} utilizada uma matriz bin{\'a}ria contendo a correla{\c c}{\~a}o entre as cargas. Esta matriz tem a hierarquia referente {\`a} topologia da rede, bem como fatores f{\'\i}sicos adicionais que s{\~a}o monitorados. Esta parte do modelo considerando as correla{\c c}{\~o}es n{\~a}o topol{\'o}gicas n{\~a}o ser{\'a} simulada, apenas conceituada e integra sugest{\~a}o para trabalhos futuros.

\subsection{Abrang{\^e}ncia} \label{subsec:f2}

\begin{equation} \label{eq:f2}
	f_{2} \left( x, P \right) = \frac{k}{N}
\end{equation}

Onde,

\begin{itemize}
	\item[] $ k $ {\'e} o total de unidades de carga descartadas em $x$, considerando os reflexos correspondentes; representa o n{\'u}mero de elementos n{\~a}o nulos em $\vec{\Pi}$, que {\'e} um vetor bin{\'a}rio. Portanto, $k = \left< \vec{\Pi}, \vec{\Pi} \right>$;
	\item[] $ N $ {\'e} o n{\'u}mero de elementos em $x$.
\end{itemize}

\subsection{Crit{\'e}rios de Opera{\c c}{\~a}o} \label{subsec:f3}

\begin{equation} \label{eq:f3}
	f_{3} \left( x, P \right) = diag\left\{ \vec{\Pi} \times \vec{crit}^{T} \right\}
\end{equation}

Onde,

\begin{itemize}
	\item[] $ \vec{crit} = \left( \frac{\sum_{\vec{crit_{0}}}^{\vec{crit_{j}}}{p_{i} \times \vec{crit_{i}}}}{j}\right) $;
	\item[] $ p_{i}$ {\'e} o fator de pondera{\c c}{\~a}o do $i-esimo$ crit{\'e}rio;
	\item[] $ \vec{crit_{i}}$ {\'e} o $i-esimo$ crit{\'e}rio.
\end{itemize}

Este termo representa a inclus{\~a}o dos crit{\'e}rios de prioridade definidos pelo operador do sistema. Assim, uma lista predefinida pelo projetista (que pode ser flexibilizada) disp{\~o}e de uma s{\'e}rie de crit{\'e}rios como, por exemplo, o painel que alimenta uma {\'a}rea f{\'\i}sica espec{\'\i}fica ou um equipamento, colocando a import{\^a}ncia das demais cargas no atendimento {\`a} esta {\'a}rea, ou, ainda, crit{\'e}rios como seguran{\c c}a. A combina{\c c}{\~a}o ponderada desses fatores permite que crit{\'e}rios secund{\'a}rios possam servir como crit{\'e}rio de desempate entre equipamentos equivalentes. Assim, duas bombas com caracter{\'\i}sticas iguais operando em paralelo no mesmo sistema, ao serem classificadas, a que apresentar maior probabilidade de falha mec{\^a}nica pode ser desligada primeiro, aumentando as chances de preserva{\c c}{\~a}o do restante do sistema.

\subsection{Ordena{\c c}{\~a}o} \label{subsec:f4}

\begin{equation} \label{eq:f4}
	f_{4} \left( x, P \right) = \sum_{i=1}^{N}{\left(N-i \right) \times crit_{x_{i}}}
\end{equation}

Onde,

\begin{itemize}
    \item $crit_{x_{i}}$ {\'e} o valor do crit{\'e}rio de opera{\c c}{\~a}o da carga na posi{\c c}{\~a}o $i$ em $x$.
\end{itemize}

A influ{\^e}ncia deste termo deve ser a menor poss{\'\i}vel, servindo apenas como crit{\'e}rio de desempate. Considerando a atua{\c c}{\~a}o por etapas, h{\'a} a possibilidade de n{\~a}o ocorrer a rejei{\c c}{\~a}o de todas as cargas previstas na tabela, de forma que as {\'u}ltimas possam ser preservadas. Portanto, h{\'a} interesse em que cargas menos priorit{\'a}rias sejam descartadas primeiro. Este termo aplica uma pondera{\c c}{\~a}o a mais aos crit{\'e}rios do termo anterior em fun{\c c}{\~a}o da ordem que cada elemento ocupa na tabela, gerando uma penalidade extra. Assim, um peso maior ter seu valor aumentado no in{\'\i}cio da tabela.

\subsection{Fun{\c c}{\~a}o Objetivo Completa} \label{subsec:obj}

\begin{equation} \label{eq:obj}
    f\left( x \right) =
    \sum_{j=1}^{5}\left(1,2-0,2\times j\right) \times
    \left\{
        \begin{matrix}
            C_{1} \times \frac{1}{e^{2}}\frac{\left( P_{j} - \left< \vec{P_{x}}, \vec{\Pi_{j}} \right> \right)^{2}}{P_{j}^{2}} & + \\
            & \\
            C_{2} \times \frac{k_{j}}{N} & + \\
            & \\
            C_{3} \times diag\left\{ \vec{\Pi_{j}} \times \vec{crit}^{T} \right\} & + \\
            & \\
            C_{4} \times \sum_{i=1}^{N}{\left(N-i \right) \times crit_{x_{i}}}
        \end{matrix}
    \right.
\end{equation}

A Equa{\c c}{\~a}o~\ref{eq:obj} {\'e} a forma final da Fun{\c c}{\~a}o Objetivo, considerando os termos definidos acima.

\section{Solu{\c c}{\~a}o Inicial} \label{sec:cldir}

Ao modificar a topologia da rede, atrav{\'e}s da abertura ou fechamento de disjuntores ou chaves seccionadoras, ou seja, ao ligar ou desligar uma carga ou painel, um novo cen{\'a}rio {\'e} gerado. Para fornecer uma solu{\c c}{\~a}o inicial de boa qualidade, utiliza-se a classifica{\c c}{\~a}o direta, utilizando como {\'\i}ndice para tal os crit{\'e}rios de opera{\c c}{\~a}o ($\vec{crit}$) da Equa{\c c}{\~a}o~\ref{eq:f3}. Esta classifica{\c c}{\~a}o servir{\'a} como solu{\c c}{\~a}o inicial para a etapa de busca, devendo, ent{\~a}o, ser otimizada.

\chapter{Simulador Para Rejei{\c c}{\~a}o de Carga} \label{cap:impl}

Para demonstrar  a efetividade do esquema para rejei{\c c}{\~a}o de cargas proposto neste trabalho, foi constru{\'\i}do um simulador digital de sistemas em tempo real, cujas telas de ingresso encontram-se nas Figuras~\ref{fig:sim} a \ref{fig:sim_pumps}. Detalhes do \textit{software} desenvolvido est{\~a}o no Ap{\^e}ndice~\ref{apend:estr}. Trata-se de um simulador que fornece uma \textbf{GUI} (\textit{Graphical User Interface}) que permite ao usu{\'a}rio montar sua pr{\'o}pria rede de testes.

A Figura~\ref{fig:sim_main} apresenta a tela do simulador que cont{\'e}m o painel el{\'e}trico principal, onde os geradores est{\~a}o situados. As Figuras~\ref{fig:sim_panels} e \ref{fig:sim_sub} apresentam telas com cargas que representam pain{\'e}is de distribui{\c c}{\~a}o, enquanto a Figura~\ref{fig:sim_pumps} apresenta cargas que representam alguns motores el{\'e}tricos das bombas industriais. Para ajustar o funcionamento, v{\^e}-se, na Figura~\ref{fig:sim_setpar}, a interface para ajustar os par{\^a}metros da fun{\c c}{\~a}o objetivo, e na Figura~\ref{fig:sim_setls}, as configura{\c c}{\~o}es do sistema de rejei{\c c}{\~a}o autom{\'a}tica de cargas, incluindo tamb{\'e}m o intervalo de atualiza{\c c}{\~a}o das leituras. Esse tempo define o tempo m{\'a}ximo dispon{\'i}vel para cada ciclo de buscas e, durante uma simula{\c c}{\~a}o, ao ser aumentado, permite verificar o desempenho equivalente de um computador com maior capacidade de c{\'a}lculo.

Utilizando a rede configurada pelo usu{\'a}rio, o \textit{software} simula um comportamento din{\^a}mico para a rede, variando aleatoriamente o valor das cargas e distribuindo a demanda pelos geradores ligados ao barramento principal. Na Figura~\ref{fig:sim_main} v{\^e}-se, {\`a} direita, a tabela de desligamento na forma de lista, dividida por etapas de desligamento, onde encontra-se destacado em escala de cores, os elementos que constituem cada etapa. Para tornar essa visualiza{\c c}{\~a}o mais simples, na tela de opera{\c c}{\~a}o, o equipamento que figura em uma das listas tem seu nome destacado com a cor respectiva desta. O sistema tamb{\'e}m permite configurar, atrav{\'e}s da tela apresentada na Figura~\ref{fig:sim_setls} o tempo entre cada ciclo de leituras, sendo este, o tempo de discretiza{\c c}{\~a}o.

Cada rede configurada pelo usu{\'a}rio pode ser armazenada em arquivo para testes posteriores. Assim, para cada arquivo salvo em disco, ao efetuar uma simula{\c c}{\~a}o, o programa gera um \textit{log} contendo os dados b{\'a}sicos para cada cen{\'a}rio gerado e seu respectivo resultado obtido. Este \textit{log} permite uma an{\'a}lise posterior de desempenho, pois com os dados do cen{\'a}rio e de configura{\c c}{\~a}o das constantes de ajuste, um programa sem limite de tempo pode procurar o melhor resultado na ``for{\c c}a bruta'' e compar{\'a}-lo ao resultado obtido em ``tempo real'' pelo algoritmo de busca.

O simulador foi escrito em linguagem de programa{\c c}{\~a}o C++, sendo a interface gr{\'a}fica escrita atrav{\'e}s da \textbf{API} (\textit{Application Programming Interface})\footnote{Interface de programa{\c c}{\~a}o de aplica{\c c}{\~o}es} QtC++\footnote{\url{http://www.qt.io}}. O programa para an{\'a}lise e apresenta{\c c}{\~a}o dos resultados obtidos foi escrito em linguagem Python\footnote{\url{http://www.python.org}}, atrav{\'e}s da interface Jupyter Notebook\footnote{\url{http://www.jupyter.org}}, permitindo mesclar texto, c{\'o}digo, gr{\'a}ficos e leitura/grava{\c c}{\~a}o em arquivo, resultando em uma boa apresenta{\c c}{\~a}o para an{\'a}lise cr{\'\i}tica do simulador. O arquivo contendo a rede constru{\'\i}da pelo usu{\'a}rio foi estruturado em \textbf{JSON} (\textit{JavaScript Object Notation})\footnote{\url{http://www.json.org}} com escrita bin{\'a}ria e extens{\~a}o customizada para o programa, sendo este um formato compacto e de f{\'a}cil manupula{\c c}{\~a}o, ocupando pouco espa{\c c}o em disco. O \textit{log} de opera{\c c}{\~a}o foi definido no formato \textbf{JSON} textual, tornando mais f{\'a}cil a importa{\c c}{\~a}o e tratamento em Python.

Considerando que o escopo deste trabalho n{\~a}o engloba a simula{\c c}{\~a}o detalhada de elementos de din{\^a}mica e controle de sistemas de gera{\c c}{\~a}o, o algoritmo utilizado para fazer o balanceamento de carga\footnote{Tamb{\'e}m conhecido pelo nome em ingl{\^e}s: \textit{Load Sharing}.} foi o mais simples poss{\'\i}vel, fazendo com que os incrementos positivos sejam absorvidos pelo gerador menos carregado percentualmente, enquanto os decr{\'e}scimos, pelos mais carregados, limitando os degraus de acelera{\c c}{\~a}o e desacelera{\c c}{\~a}o {\`a} in{\'e}rcia pr{\'o}pria de cada gerador (constante $H$ definida pela Equa{\c c}{\~a}o~\ref{eq:hinerc}). A tend{\^e}ncia a longo prazo {\'e} o equil{\'\i}brio na carga percentual das m{\'a}quinas. Apesar de simples, esta configura{\c c}{\~a}o {\'e} bastante precisa e veross{\'\i}mil. Todavia, a forma como o simulador foi constru{\'\i}do torna vi{\'a}vel uma melhoria futura, visando implementar algoritmos mais sofisticados para controle de gera{\c c}{\~a}o.

Da mesma forma que o algoritmo de simula{\c c}{\~a}o da gera{\c c}{\~a}o, um algoritmo simplificado simula o comportamento aleat{\'o}rio das cargas el{\'e}tricas. Esta {\'e} a raz{\~a}o dos experimentos realizados aqui serem {\'u}nicos e n{\~a}o serem pass{\'i}veis de reprodu{\c c}{\~a}o exata, j{\'a} que a mesma sequ{\^e}ncia de opera{\c c}{\~o}es, considerando a aleatoriedade de comportamento, leva a resultados distintos. Para definir este comportamento, as cargas foram divididas em duas classes: cargas est{\'a}ticas e cargas din{\^a}micas. As cargas est{\'a}ticas representam, basicamente, pain{\'e}is de distribui{\c c}{\~a}o locais, contendo ilumina{\c c}{\~a}o, tomadas e cargas menores diversas. Estas cargas partem essencialmente desligadas e o algoritmo considera uma varia{\c c}{\~a}o que tende a diminuir quando a carga se aproxima de seu valor nominal, tendendo, assim, a estabilizar a longo prazo. J{\'a} as cargas din{\^a}micas representam os motores de grande porte de unidades industriais. Como caracter{\'\i}stica principal, os motores demandam alta pot{\^e}ncia na partida, assim, o comportamento foi ajustado para estabilizar em torno de $90\%$ da pot{\^e}ncia nominal, e, mesmo apresentando alguma varia{\c c}{\~a}o, tende sempre a retornar a este valor. A velocidade de aumento ou redu{\c c}{\~a}o de carga para motores est{\'a} limitada {\`a} sua in{\'e}rcia mec{\^a}nica associada.

Da mesma forma que a implementa{\c c}{\~a}o do algoritmo de controle de gera{\c c}{\~a}o permite melhoria, este algoritmo de simula{\c c}{\~a}o de carga foi escrito de forma modular, classificado por {\'\i}ndices internos, sendo f{\'a}cil, portanto, a implementa{\c c}{\~a}o de altera{\c c}{\~o}es, tanto da filosofia de simula{\c c}{\~a}o, quanto a insers{\~a}o de novos tipos de carga, com comportamentos distintos. Assim, uma vers{\~a}o mais elaborada pode trabalhar, por exemplo, com o modelo \textbf{ZIP} \cite{guimaraessistemas} para uma representa{\c c}{\~a}o de cargas mais realista, principalmente para estender a cargas n{\~a}o industriais, como sistemas de distribui{\c c}{\~a}o urbana.

\begin{figure}
	\centering
	\includegraphics[width=\linewidth]{figuras/simulator}
	\caption[Simulador de Rejei{\c c}{\~a}o de Cargas \---  Tela Inicial]{Simulador de Rejei{\c c}{\~a}o de Cargas \--- Tela Inicial [Fonte: acervo pessoal]}
	\label{fig:sim}
\end{figure}

\begin{figure}
	\centering
	\includegraphics[width=0.8\linewidth]{figuras/parameters}
	\caption[Simulador de Rejei{\c c}{\~a}o de Cargas \--- Par{\^a}metros da Fun{\c c}{\~a}o Objetivo]{Simulador de Rejei{\c c}{\~a}o de Cargas \--- Par{\^a}metros da Fun{\c c}{\~a}o Objetivo [Fonte: acervo pessoal]}
	\label{fig:sim_setpar}
\end{figure}

\begin{figure}
	\centering
	\includegraphics[width=0.8\linewidth]{figuras/settings}
	\caption[Simulador de Rejei{\c c}{\~a}o de Cargas \--- Par{\^a}metros ddo \textbf{UFLS}]{Simulador de Rejei{\c c}{\~a}o de Cargas \--- Par{\^a}metros do \textbf{UFLS} [Fonte: acervo pessoal]}
	\label{fig:sim_setls}
\end{figure}

\begin{figure}
	\centering
	\includegraphics[width=\linewidth]{figuras/simulator_main}
	\caption[Simulador de Rejei{\c c}{\~a}o de Cargas \--- Geradores]{Simulador de Rejei{\c c}{\~a}o de Cargas \--- Geradores [Fonte: acervo pessoal]}
	\label{fig:sim_main}
\end{figure}

\begin{figure}
	\centering
	\includegraphics[width=\linewidth]{figuras/simulator_panels}
	\caption[Simulador de Rejei{\c c}{\~a}o de Cargas \--- Pain{\'e}is]{Simulador de Rejei{\c c}{\~a}o de Cargas \--- Pain{\'e}is [Fonte: acervo pessoal]}
	\label{fig:sim_panels}
\end{figure}

\begin{figure}
	\centering
	\includegraphics[width=\linewidth]{figuras/simulator_subpanels}
	\caption[Simulador de Rejei{\c c}{\~a}o de Cargas \--- Sub-Pain{\'e}is]{Simulador de Rejei{\c c}{\~a}o de Cargas \--- Sub-Pain{\'e}is [Fonte: acervo pessoal]}
	\label{fig:sim_sub}
\end{figure}

\begin{figure}
	\centering
	\includegraphics[width=\linewidth]{figuras/simulator_pumps}
	\caption[Simulador de Rejei{\c c}{\~a}o de Cargas \--- Motores de Bombas]{Simulador de Rejei{\c c}{\~a}o de Cargas \--- Motores de Bombas [Fonte: acervo pessoal]}
	\label{fig:sim_pumps}
\end{figure}

\chapter{Resultados} \label{cap:teste}

\section{Planta Utilizada Para Testes} \label{sec:rede}

Para a realiza{\c c}{\~a}o de testes, foi configurada uma rede fict{\'\i}cia, t{\'\i}pica de plataforma de produ{\c c}{\~a}o de petr{\'o}leo em regime \textit{off shore}\footnote{Baseada na experi{\^e}ncia profissional do autor deste trabalho.}. Esta rede cont{\'e}m duas turbinas de gera{\c c}{\~a}o a g{\'a}s natural, com capacidade nominal de $6MW$ de pot{\^e}ncia cada e quatro geradores auxiliares com capacidade nominal de $2MW$ cada, conforme indicado na Tabela~\ref{tab:gen}. Esta configura{\c c}{\~a}o {\'e} comum, pois, em situa{\c c}{\~a}o normal, as plataformas de petr{\'o}leo disp{\~o}em de g{\'a}s oriundo da pr{\'o}pria produ{\c c}{\~a}o para manter as turbinas em opera{\c c}{\~a}o. Em situa{\c c}{\~o}es at{\'\i}picas, como manuten{\c c}{\~a}o ou indisponibilidade de uma turbina, a gera{\c c}{\~a}o faltante {\'e} compensada pela entrada dos geradores auxiliares. Em caso de paradas de produ{\c c}{\~a}o, a plataforma deve se manter utilizando {\'o}leo diesel estocado em tanques, sendo alimentada somente pelos geradores auxiliares\footnote{Em alguns casos, as turbinas a g{\'a}s s{\~a}o bi-combust{\'\i}veis, operando tanto com g{\'a}s quanto diesel.}.

Os geradores s{\~a}o conectados no barramento do painel principal, {\`a}s vezes referido como painel de gera{\c c}{\~a}o ou Subesta{\c c}{\~a}o Principal. Este painel alimenta toda a unidade, atrav{\'e}s de outros pain{\'e}is, conforme Tabela~\ref{tab:loadp1}. As cargas, que consistem em motores, Centros de Controle de Motor \--- CCMs, pain{\'e}is de distribui{\c c}{\~a}o, entre outras, est{\~a}o discriminadas nas Tabelas~\ref{tab:loadp2} a \ref{tab:loadp6}.

\begin{table}[!h]
	\begin{center}
		\caption{Lista de Geradores na Subesta{\c c}{\~a}o Principal}
		\label{tab:gen}
	    \vspace{5pt}
		\begin{tabular}{c c c c}
			\hline
			\textbf{Identifica{\c c}{\~a}o} & \textbf{Pot{\^e}ncia} & \textbf{For{\c c}a Motriz} & \textbf{Barra}\\
			\hline\hline
			Turbogerador A & $6MW$ & Turbina a G{\'a}s & \\
			Motogerador A & $2MW$ & Motor a Diesel & \textbf{A} \\
			Motogerador C & $2MW$ & Motor a Diesel & \\
			\hline\hline
			Motogerador B & $2MW$ & Motor a Diesel & \\
			Motogerador D & $2MW$ & Motor a Diesel & \textbf{B} \\
			Turbogerador B & $6MW$ & Turbina a G{\'a}s & \\
			\hline
		\end{tabular}
	\end{center}
\end{table}

\begin{table}[!h]
	\begin{center}
		\caption{Lista de Cargas na Subesta{\c c}{\~a}o Principal}
		\label{tab:loadp1}
	    \vspace{5pt}
		\begin{tabular}{c c c c}
			\hline
			\textbf{Identifica{\c c}{\~a}o} & \textbf{Pot{\^e}ncia} & \textbf{Descri{\c c}{\~a}o} & \textbf{Barra} \\
			\hline\hline
			Bombas Produ{\c c}{\~a}o \--- A & $5490kW$ & Produ{\c c}{\~a}o e Transfer{\^e}ncia de {\'O}leo & \\
			CCMs Produ{\c c}{\~a}o \--- A & $580kW$ & Auxiliares de Produ{\c c}{\~a}o & \textbf{A} \\
			Distribui{\c c}{\~a}o \--- A & $1460,53kW$ & Demais Cargas & \\
			\hline\hline
			Bombas Produ{\c c}{\~a}o \--- B & $5471kW$ & Produ{\c c}{\~a}o e Transfer{\^e}ncia de {\'O}leo & \\
			CCMs Produ{\c c}{\~a}o \--- B & $580kW$ & Auxiliares de Produ{\c c}{\~a}o & \textbf{B} \\
			Distribui{\c c}{\~a}o \--- B & $1460,53kW$ & Demais Cargas & \\
			\hline
		\end{tabular}
	\end{center}
\end{table}

\begin{table}[!h]
	\begin{center}
		\caption{Lista de Cargas no Painel Bombas de Produ{\c c}{\~a}o \--- A}
		\label{tab:loadp2}
	    \vspace{5pt}
		\begin{tabular}{c c c}
			\hline
			\textbf{Identifica{\c c}{\~a}o} & \textbf{Pot{\^e}ncia} & \textbf{Descri{\c c}{\~a}o}\\
			\hline\hline
			Capta{\c c}{\~a}o de {\'A}gua \--- A & $130kW$ & Bomba de Capta{\c c}{\~a}o de {\'A}gua do Mar \\
			Capta{\c c}{\~a}o de {\'A}gua \--- C & $130kW$ & Bomba de Capta{\c c}{\~a}o de {\'A}gua do Mar \\
			Capta{\c c}{\~a}o de {\'A}gua \--- E & $130kW$ & Bomba de Capta{\c c}{\~a}o de {\'A}gua do Mar \\
			Exporta{\c c}{\~a}o de {\'O}leo \--- A & $900kW$ & Bomba Principal de Exporta{\c c}{\~a}o \\
			Exporta{\c c}{\~a}o de {\'O}leo \--- C & $900kW$ & Bomba Principal de Exporta{\c c}{\~a}o \\
			Inje{\c c}{\~a}o de {\'A}gua \--- A & $1100kW$ & Bomba Principal de Inje{\c c}{\~a}o de {\'A}gua \\
			Inje{\c c}{\~a}o de {\'A}gua \--- C & $1100kW$ & Bomba Principal de Inje{\c c}{\~a}o de {\'A}gua \\
			Inje{\c c}{\~a}o de {\'A}gua \--- E & $1100kW$ & Bomba Principal de Inje{\c c}{\~a}o de {\'A}gua \\
			\hline
		\end{tabular}
	\end{center}
\end{table}

\begin{table}[!h]
	\begin{center}
		\caption{Lista de Cargas no Painel Bombas de Produ{\c c}{\~a}o \--- B}
		\label{tab:loadp3}
	    \vspace{5pt}
		\begin{tabular}{c c c}
			\hline
			\textbf{Identifica{\c c}{\~a}o} & \textbf{Pot{\^e}ncia} & \textbf{Descri{\c c}{\~a}o}\\
			\hline\hline
			Capta{\c c}{\~a}o de {\'A}gua \--- B & $130kW$ & Bomba de Capta{\c c}{\~a}o de {\'A}gua do Mar \\
			Capta{\c c}{\~a}o de {\'A}gua \--- D & $130kW$ & Bomba de Capta{\c c}{\~a}o de {\'A}gua do Mar \\
			Capta{\c c}{\~a}o de {\'A}gua \--- F & $130kW$ & Bomba de Capta{\c c}{\~a}o de {\'A}gua do Mar \\
			Bomba de Inc{\^e}ndio & $410kW$ & Bomba de Combate a Inc{\^e}ndio \\
			Compressor Booster de G{\'a}s & $671kW$ & Compressor de Recupera{\c c}{\~a}o de G{\'a}s \\
			Exporta{\c c}{\~a}o de {\'O}leo \--- B & $900kW$ & Bomba Principal de Exporta{\c c}{\~a}o \\
			Exporta{\c c}{\~a}o de {\'O}leo \--- D & $900kW$ & Bomba Principal de Exporta{\c c}{\~a}o \\
			Inje{\c c}{\~a}o de {\'A}gua \--- B & $1100kW$ & Bomba Principal de Inje{\c c}{\~a}o de {\'A}gua \\
			Inje{\c c}{\~a}o de {\'A}gua \--- D & $1100kW$ & Bomba Principal de Inje{\c c}{\~a}o de {\'A}gua \\
			\hline
		\end{tabular}
	\end{center}
\end{table}

\begin{table}[!h]
	\begin{center}
		\caption{Lista de Cargas no Painel CCMs Produ{\c c}{\~a}o}
		\label{tab:loadp4}
	    \vspace{5pt}
		\begin{tabular}{c c c c}
			\hline
			\textbf{Identifica{\c c}{\~a}o} & \textbf{Pot{\^e}ncia} & \textbf{Descri{\c c}{\~a}o} & \textbf{Barra} \\
			\hline\hline
			CCM Inje{\c c}{\~a}o \--- A & $300kW$ & Controle das Bombas de Inje{\c c}{\~a}o & \\
			Turbocompressor A & $50kW$ & Controle do Compressor de G{\'a}s & \textbf{A} \\
			Turbocompressor C & $50kW$ & Controle do Compressor de G{\'a}s & \\
			\hline\hline
			CCM Inje{\c c}{\~a}o \--- B & $300kW$ & Controle das Bombas de Inje{\c c}{\~a}o & \\
			Turbocompressor B & $50kW$ & Controle do Compressor de G{\'a}s & \textbf{B} \\
			Unidade Qu{\'\i}mica \--- A & $120kW$ & Inje{\c c}{\~a}o de Produtos Qu{\'\i}micos & \\
			Unidade Qu{\'\i}mica \--- B & $120kW$ & Inje{\c c}{\~a}o de Produtos Qu{\'\i}micos & \\
			\hline
		\end{tabular}
	\end{center}
\end{table}

\begin{table}[!h]
	\begin{center}
		\caption{Lista de Cargas no Painel de Distribui{\c c}{\~a}o}
		\label{tab:loadp5}
	    \vspace{5pt}
		\begin{tabular}{c c c c}
			\hline
			\textbf{Identifica{\c c}{\~a}o} & \textbf{Pot{\^e}ncia} & \textbf{Descri{\c c}{\~a}o} & \textbf{Barra} \\
			\hline\hline
			Ventila{\c c}{\~a}o \--- A & $584,49kW$ & Ventila{\c c}{\~a}o e Ar Condicionado & \\
			Cargas Facilidades \--- A & $185,90kW$ & Facilidades N{\~a}o El{\'e}tricas & \\
			Bombas Facilidades \--- A & $373,50kW$ & Bombas Gerais & \textbf{A} \\
			Ilumina{\c c}{\~a}o e Tomadas \--- A & $180,80kW$ & Ilumina{\c c}{\~a}o e Tomadas & \\
			 Uso Geral \--- A & $30kW$ & Cargas Gerais e Tomadas & \\
			Emerg{\^e}ncia \--- A & $30kW$ & Cargas Essenciais & \\
			\hline\hline
			Ventila{\c c}{\~a}o \--- B & $584,49kW$ & Ventila{\c c}{\~a}o e Ar Condicionado & \\
			Cargas Facilidades \--- B & $151,83kW$ & Facilidades N{\~a}o El{\'e}tricas & \\
			Bombas Facilidades \--- B & $610,15kW$ & Bombas Gerais & \textbf{B} \\
			Ilumina{\c c}{\~a}o e Tomadas \--- B & $250kW$ & Ilumina{\c c}{\~a}o e Tomadas & \\
			Uso Geral \--- B & $30kW$ & Cargas Gerais e Tomadas & \\
			Emerg{\^e}ncia \--- B & $30kW$ & Cargas Essenciais & \\
			\hline
		\end{tabular}
	\end{center}
\end{table}

\begin{table}[!h]
	\begin{center}
		\caption{Lista de Cargas no Painel de Emerg{\^e}ncia}
		\label{tab:loadp6}
	    \vspace{5pt}
		\begin{tabular}{c c c c}
			\hline
			\textbf{Identifica{\c c}{\~a}o} & \textbf{Pot{\^e}ncia} & \textbf{Descri{\c c}{\~a}o} & \textbf{Barra} \\
			\hline\hline
			CCM Essenciais \--- A & $316,49kW$ & Controle de Bombas Essenciais & \\
			Gerador de Emerg{\^e}ncia \--- A & $40kW$ & Controle do Gerador A & \textbf{A} \\
			Turbogerador A & $56kW$ & Controle do Turbogerador A & \\
			\hline\hline
			CCM Essenciais \--- B & $415,47kW$ & Controle de Bombas Essenciais & \\
			Gerador de Emerg{\^e}ncia \--- B & $40kW$ & Controle do Gerador B & \textbf{B} \\
			Turbogerador B & $56kW$ & Controle do Turbogerador B & \\
			\hline
		\end{tabular}
	\end{center}
\end{table}

\section{Crit{\'e}rios Adotados}

A partir do sum{\'a}rio de testes descrito na Se{\c c}{\~a}o~\ref{sec:rede}, algumas varia{\c c}{\~o}es foram inseridas nos ajustes das constantes da Equa{\c c}{\~a}o~\ref{eq:obj}, a saber, $C_{1}$, $C_{2}$, $C_{3}$ e $C_{4}$, objetivando comparar a influ{\^e}ncia destas sobre os resultados e, atrav{\'e}s de tentativa, buscar um padr{\~a}o de otimiza{\c c}{\~a}o, tanto dos resultados, quanto da velocidade de obten{\c c}{\~a}o na busca iterada. Entretanto, como cada par{\^a}metro no simulador foi programado com um intervalo de ajuste entre 0 e 99, h{\'a} 100 milh{\~o}es de combina{\c c}{\~o}es poss{\'\i}veis. Portanto, para tornar vi{\'a}vel a realiza{\c c}{\~a}o dos testes, foram selecionados 25 combina{\c c}{\~o}es de valores levando em conta os seguintes crit{\'e}rios:

\begin{itemize}
    \item os valores foram selecionados valores para as constantes dentro do conjunto (0, 25, 50, 99), de modo que a soma das constantes estivesse contida no conjunto (99, 198, 297, 396, 124, 149);
    \item a combina{\c c}{\~a}o deve priorizar as rela{\c c}{\~o}es qualitativas entre os par{\^a}metros para permitir compara{\c c}{\~a}o;
    \item deve conter um n{\'u}mero limitado de casos representativos, tornando a simula{\c c}{\~a}o vi{\'a}vel.
\end{itemize}

Para selecionar estes valores, foi utilizado um  algoritmo que seleciona as combina{\c c}{\~o}es de par{\^a}metros de acordo com a soma, comparando a um conjunto previamente estabelecido de valores. Assim, foram selecionados os conjuntos de par{\^a}metros apresentados na Tabela~\ref{tab:cenarios}. Estes valores foram utilizados para configurar 25 cen{\'a}rios de teste, utilizando a mesma rede, no mesmo contexto, comparando apenas as caracter{\'\i}sticas da solu{\c c}{\~a}o obtida.

\begin{table}[!b]
	\begin{center}
		\caption[Configura{\c c}{\~o}es dos Cen{\'a}rios de Teste \--- Par{\^a}metros Utilizados]{Configura{\c c}{\~o}es dos Cen{\'a}rios de Teste \--- Par{\^a}metros Utilizados [Fonte: Algoritmo Pr{\'o}prio]}
		\label{tab:cenarios}
		\vspace{5pt}
		\begin{tabular}{c c c c c}
			\hline
			\textbf{Nome} & \textbf{$C_{1}$} & \textbf{$C_{2}$} & \textbf{$C_{3}$} & \textbf{$C_{4}$} \\
			\hline\hline
			Teste 01 & 0 & 0 & 0 & 99 \\
			Teste 02 & 0 & 0 & 25 & 99 \\
			Teste 03 & 0 & 0 & 50 & 99 \\
			Teste 04 & 0 & 0 & 99 & 25 \\
			Teste 05 & 0 & 0 & 99 & 50 \\
			Teste 06 & 0 & 25 & 0 & 99 \\
			Teste 07 & 0 & 25 & 25 & 99 \\
			Teste 08 & 0 & 25 & 99 & 0 \\
			Teste 09 & 0 & 25 & 99 & 25 \\
			Teste 10 & 0 & 99 & 0 & 25 \\
			Teste 11 & 0 & 99 & 0 & 50 \\
			Teste 12 & 0 & 99 & 25 & 0 \\
			Teste 13 & 0 & 99 & 25 & 25 \\
			Teste 14 & 25 & 0 & 0 & 99 \\
			Teste 15 & 25 & 0 & 25 & 99 \\
			Teste 16 & 25 & 0 & 99 & 0 \\
			Teste 17 & 25 & 0 & 99 & 25 \\
			Teste 18 & 25 & 99 & 0 & 0 \\
			Teste 19 & 25 & 99 & 0 & 25 \\
			Teste 20 & 99 & 0 & 0 & 25 \\
			Teste 21 & 99 & 0 & 0 & 50 \\
			Teste 22 & 99 & 0 & 25 & 0 \\
			Teste 23 & 99 & 0 & 25 & 25 \\
			Teste 24 & 99 & 25 & 0 & 0 \\
			Teste 25 & 99 & 25 & 0 & 25 \\
			\hline
		\end{tabular}
	\end{center}
\end{table}

As an{\'a}lises dos testes realizados foram feitas utilizando o hist{\'o}rico de opera{\c c}{\~a}o gerado individualmente. Neste hist{\'o}rico s{\~a}o fornecidas informa{\c c}{\~o}es instant{\^a}neas a cada leitura na discretiza{\c c}{\~a}o da simula{\c c}{\~a}o. S{\~a}o estas informa{\c c}{\~o}es:

\begin{itemize}
    \item nomes das cargas na ordem que o sistema gera;
    \item valores instant{\^a}neos de pot{\^e}ncia el{\^e}trica;
    \item valores das constantes de ajuste do sistema;
    \item valores dos crit{\'e}rios de opera{\c c}{\~a}o utilizados;
    \item custo obtido na solu{\c c}{\~a}o corrente no in{\'\i}cio do ciclo de busca;
    \item custo obtido na solu{\c c}{\~a}o atualizada ao final do ciclo de busca;
    \item percentuais de ajuste das etapas de desligamento.
\end{itemize}

Os valores de custo, inicial e final, de cada per{\'\i}odo de discretiza{\c c}{\~a}o permitem criar um hist{\'o}rico com os ganhos obtidos na busca ao longo do tempo, considerando que cada ciclo de busca {\'e} realizado dentro do intervalo de discretiza{\c c}{\~a}o\footnote{Nos testes realizados, este intervalo foi de $500ms$.}. Este ganho, chamado $gap$, foi calculado pela seguinte Equa{\c c}{\~a}o:

\begin{equation} \label{eq:gap}
    gap = \frac{f\left( x_{i} \right) - f\left( x_{o} \right)}{f\left( x_{i} \right)} \times 100\%
\end{equation}

Onde,

\begin{itemize}
    \item[] $gap$ {\'e} o percentual de ganho do ciclo de busca;
    \item[] $f\left( x_{i} \right)$ {\'e} o custo da solu{\c c}{\~a}o no in{\'\i}cio do ciclo;
    \item[] $f\left( x_{o} \right)$ {\'e} o custo da solu{\c c}{\~a}o ao final.
\end{itemize}

\section{Resultados Obtidos} \label{sec:result}

A partir da Equa{\c c}{\~a}o~\ref{eq:gap}, foi gerado um hist{\'o}rico de $gap$ para cada cen{\'a}rio de teste. Para ilustrar como os resultados foram tratados, ser{\'a} visto o cen{\'a}rio de teste 12 e sua respectiva an{\'a}lise.

A Figura~\ref{fig:Teste_12.hist} apresenta o hist{\'o}rico de $gap$ obtido no Teste 12 na ordem que que ocorreram ao longo do teste, bem como o mesmo hist{\'o}rico reclassificado em valores decrescentes. Esta {\'u}ltima vizualiza{\c c}{\~a}o permite um melhor entendimento de como os valores se distribuem em torno da m{\'e}dia.

\begin{figure}[!h]
	\centering
	\includegraphics[width=1.0\linewidth]{figuras/hist}
	\caption[Teste 12 \--- Hist{\'o}rico de $gap$]{Teste 12 \--- Hist{\'o}rico de Gap [Fonte: Simula{\c c}{\~a}o]}
	\label{fig:Teste_12.hist}
\end{figure}

Portanto, a partir da visualiza{\c c}{\~a}o, torna-se evidente a discrep{\^a}ncia entre os valores que encontram-se acima e abaixo da m{\'e}dia. Ocorre que os valores que est{\~a}o abaixo da m{\'e}dia apresentam baixos valores de $gap$, pois j{\'a} encontram-se acomodados em torno de uma solu{\c c}{\~a}o {\'o}tima local, e apresentam apenas pequenos ajustes de acordo com a varia{\c c}{\~a}o nas cargas da rede. J{\'a} os valores mais altos ocorrem quando uma nova solu{\c c}{\~a}o inicial {\'e} gerada (quando h{\'a} mudan{\c c}a na topologia da rede ocasionada por entrada ou sa{\'\i}da de carga ou painel), tornando necess{\'a}ria a busca de um ponto otimizado a partir da nova solu{\c c}{\~a}o. Portanto, esses valores, classificados como $gap$ ``em acomoda{\c c}{\~a}o'', s{\~a}o os valores que ilustram o quanto a solu{\c c}{\~a}o otimizada se distancia da solu{\c c}{\~a}o inicial.

A Figura~\ref{fig:Teste_12.trans} apresenta os valores de $gap$ em ordem decrescente apenas dos valores acima da m{\'e}dia, classificados como regime em acomoda{\c c}{\~a}o. A m{\'e}dia destes valores, chamada de $gap_{medio}$, ser{\'a} utilizada para compara{\c c}{\~a}o entre os resultados para inferir a influ{\^e}ncia dos par{\^a}metros de ajuste sobre o comportamento do esquema apresentado.


\begin{figure}[!h]
	\centering
	\includegraphics[width=1.0\linewidth]{figuras/trans}
	\caption[Teste 12 \--- Gap em Acomoda{\c c}{\~a}o]{Teste 12 \--- Gap em Acomoda{\c c}{\~a}o [Fonte: Simula{\c c}{\~a}o]}
	\label{fig:Teste_12.trans}
\end{figure}

A Tabela~\ref{tab:compara} apresenta os resultados obtidos nos 25 cen{\'a}rios de teste realizados.


\begin{table}[!h]
	\begin{center}
		\caption[Resumo dos Resultados \--- Par{\^a}metros Utilizados e M{\'e}dia de Ganhos no Regime de Acomoda{\c c}{\~a}o]{Resumo dos Resultados \--- Par{\^a}metros Utilizados e M{\'e}dia de Ganhos no Regime de Acomoda{\c c}{\~a}o [Fonte: Simula{\c c}{\~a}o]}
		\label{tab:compara}
		\vspace{5pt}
		\begin{tabular}{c c c c c c}
			\hline
			\textbf{Cen{\'a}rio de Teste} & \textbf{$C_{1}$} & \textbf{$C_{2}$} & \textbf{$C_{3}$} & \textbf{$C_{4}$} & \textbf{$gap_{medio}$} \\
			\hline\hline
			Teste 01 & 0 & 0 & 0 & 99 & 0.00 \\
			Teste 02 & 0 & 0 & 25 & 99 & 18.99 \\
			Teste 03 & 0 & 0 & 50 & 99 & 12.02 \\
			Teste 04 & 0 & 0 & 99 & 25 & 8.02 \\
			Teste 05 & 0 & 0 & 99 & 50 & 17.35 \\
			Teste 06 & 0 & 25 & 0 & 99 & 32.39 \\
			Teste 07 & 0 & 25 & 25 & 99 & 9.96 \\
			Teste 08 & 0 & 25 & 99 & 0 & 10.05 \\
			Teste 09 & 0 & 25 & 99 & 25 & 13.16 \\
			Teste 10 & 0 & 99 & 0 & 25 & 21.98 \\
			Teste 11 & 0 & 99 & 0 & 50 & 31.59 \\
			Teste 12 & 0 & 99 & 25 & 0 & 6.67 \\
			Teste 13 & 0 & 99 & 25 & 25 & 13.31 \\
			Teste 14 & 25 & 0 & 0 & 99 & 19.99 \\
			Teste 15 & 25 & 0 & 25 & 99 & 24.75 \\
			Teste 16 & 25 & 0 & 99 & 0 & 6.83 \\
			Teste 17 & 25 & 0 & 99 & 25 & 12.16 \\
			Teste 18 & 25 & 99 & 0 & 0 & 5.61 \\
			Teste 19 & 25 & 99 & 0 & 25 & 9.93 \\
			Teste 20 & 99 & 0 & 0 & 25 & 7.26 \\
			Teste 21 & 99 & 0 & 0 & 50 & 8.98 \\
			Teste 22 & 99 & 0 & 25 & 0 & 11.07 \\
			Teste 23 & 99 & 0 & 25 & 25 & 9.19 \\
			Teste 24 & 99 & 25 & 0 & 0 & 13.60 \\
			Teste 25 & 99 & 25 & 0 & 25 & 15.16 \\
			\hline
		\end{tabular}
	\end{center}
\end{table}

\section{An{\'a}lise dos Resultados} \label{sec:anares}

Observando o sum{\'a}rio de testes apresentado na Tabela~\ref{tab:compara}, observa-se que os dois primeiros fatores da Equa{\c c}{\~a}o~\ref{eq:obj}, quando ressaltados, trazem ganhos mais expressivos na busca, resultando em maiores altera{\c c}{\~o}es. Este comportamento {\'e} esperado, uma vez que s{\~a}o estes termos que diferem da solu{\c c}{\~a}o cl{\'a}ssica, expressa nos dois {\'u}ltimos termos. Ou seja, ao ressaltar a quantidade m{\'i}nima de cargas e a menor diferen{\c c}a de pot{\^e}ncia el{\'e}trica poss{\'\i}vel entre a carga efetivamente descartada e a carga que deve ser descartada, o sistema vai, na maioria das vezes, adequar a solu{\c c}{\~a}o predefinida para atender a estes requisitos. Assim, ganhos maiores na busca apenas ressaltam o quanto a solu{\c c}{\~a}o foi alterada sem interfer{\^e}ncia do operador, n{\~a}o fazendo, portanto, ju{\'\i}zo de valor sobre as cargas efetivamente selecionadas. Em outras palavras, n{\~a}o {\'e} um ganho absoluto, mas relativo {\`a} solu{\c c}{\~a}o anterior, que, por sua vez, pode j{\'a} ser suficientemente boa para n{\~a}o demandar muitas altera{\c c}{\~o}es.

Os testes realizados podem ser considerados satisfat{\'o}rios, pois demonstraram a versatilidade do m{\'e}todo proposto dentro dos cen{\'a}rios simulados, bem como uma tend{\^e}ncia de comportamento relativa a parametriza{\c c}{\~a}o dos ajustes.

H{\'a} uma parte da ferramenta que n{\~a}o foi explicitamente implementada na simula{\c c}{\~a}o. Por{\'e}m, os resultados obtidos corroboram sua efetividade. Trata-se da multiplicidade de crit{\'e}rios de opera{\c c}{\~a}o expressa na Equa{\c c}{\~a}o~\ref{eq:f3} que foi definida na Subsess{\~a}o~\ref{subsec:f3}. Para esta simula{\c c}{\~a}o, apenas um conjunto de crit{\'e}rios de import{\^a}ncia foi implementado, representando em si o resultado da m{\'e}dia ponderada entre os m{\'u}ltiplos cen{\'a}rios poss{\'\i}veis, ajustado para refletir a tabela est{\'a}tica da solu{\c c}{\~a}o cl{\'a}ssica. Entretanto, diversos conjuntos poderiam ser utilizados, e o valor final poderia ser uma sele{\c c}{\~a}o entre estes ou, ainda, uma combina{\c c}{\~a}o ponderada destes. 

O cen{\'a}rio Teste~01 que na meta-heur{\'\i}stica obteve $gap_{medio}$ nulo, demonstra um enorme poder de sucesso na sele{\c c}{\~a}o de m{\'u}ltiplos cen{\'a}rios. Assim, pode-se obter uma tabela est{\'a}tica (ou diretriz de solu{\c c}{\~a}o inicial) para o cen{\'a}rio de interesse, que, no caso de uma plataforma de petr{\'o}leo, poderia priorizar, por exemplo, os sistemas de produ{\c c}{\~a}o, facilidades ou seguran{\c c}a.

\chapter{Conclus{\~o}es e Trabalhos Futuros} \label{cap:concl}

Este trabalho construiu um sistema inteligente para tomada de decis{\~o}es na determina{\c c}{\~a}o de cargas a serem desligadas automaticamente, atrav{\'e}s de uma solu{\c c}{\~a}o de baixo impacto na rede, aplicando t{\'e}cnicas de intelig{\^e}ncia artificial a m{\'e}todos de atua{\c c}{\~a}o de rel{\'e}s.

Os testes realizados corroboraram a efic{\'a}cia do m{\'e}todo proposto, ainda que no campo da simula{\c c}{\~a}o, j{\'a} que testes em sistemas reais demandam um custo demasiado elevado, tornando-os invi{\'a}veis. Neste aspecto, o \textit{software} desenvolvido mostrou-se eficiente e robusto, pois, al{\'e}m de realizar uma simula{\c c}{\~a}o bastante completa do ponto de vista da din{\^a}mica do sistema de pot{\^e}ncia, tamb{\'e}m conseguiu simular os efeitos de prote{\c c}{\~a}o e atua{\c c}{\~a}o do sistema de rejei{\c c}{\~a}o autom{\'a}tica de cargas, bem como a fun{\c c}{\~a}o de busca, utilizando, para tanto, um hardware comum e acess{\'\i}vel.

Com a simula{\c c}{\~a}o realizada, foi utilizado o algoritmo de busca \textbf{VND} para fazer convergir a solu{\c c}{\~a}o para o menor custo encontrado, utilizando, para avalia{\c c}{\~a}o deste custo, a equa{\c c}{\~a}o proposta neste trabalho, que oferece ao operador a flexibilidade de definir par{\^a}metros do algoritmo para fugir da solu{\c c}{\~a}o cl{\'a}ssica, visando uma melhor adequa{\c c}{\~a}o aos requisitos el{\'e}tricos da rede.

Conforme descrito ao longo do trabalho, as solu{\c c}{\~o}es existentes s{\~a}o bastante detalhadas e consolidadas para atender ao problema do ponto de vista el{\'e}trico, e, eventualmente, econ{\^o}mico, no caso das concession{\'a}rias de distribui{\c c}{\~a}o. Entretanto, este trabalho visa combinar aspectos distintos para fornecer uma solu{\c c}{\~a}o diferenciada com m{\'u}ltiplos objetivos, aliando aspectos el{\'e}tricos e operacionais, permitindo ao operador definir o quanto a decis{\~a}o deve pender para um aspecto ou outro atrav{\'e}s dos par{\^a}metros de configura{\c c}{\~a}o.

Este trabalho foi, portanto, bem sucedido na proposta de aumentar a variedade de solu{\c c}{\~o}es dispon{\'\i}veis, agregando em si aspectos ben{\'e}ficos encontrados separadamente em outras solu{\c c}{\~o}es, sem perder a robustez computacional.

Trabalhos futuros s{\~a}o divisados, considerando que o escopo deste trabalho foi espec{\'\i}fico ao simular opera{\c c}{\~a}o de sistemas ilhados, embora o m{\'e}todo sugerido possa ser utilizado em sistemas conectados {\`a} rede el{\'e}trica p{\'u}blica, eletricamente chamada de barra infinita. Para tanto, a alimenta{\c c}{\~a}o da rede deve ser considerada como um gerador extra, cuja pot{\^e}ncia nominal seja a pot{\^e}ncia para qual a prote{\c c}{\~a}o da entrada atue por sobrecorrente. Esse t{\'o}pico configura um estudo em aberto.

Outra limita{\c c}{\~a}o considerada aqui {\'e} a gera{\c c}{\~a}o concentrada na subesta{\c c}{\~a}o principal. Este normalmente {\'e} o caso de sistemas ilhados, como plataformas de petr{\'o}leo, plantas industriais de produ{\c c}{\~a}o ou estabelecimentos comerciais que operem offline em hor{\'a}rio de ponta. Entretanto, redes que alimentem sistemas de trem ou metr{\^o}, por exemplo, podem ter a gera{\c c}{\~a}o distribu{\'\i}da por subesta{\c c}{\~a}oes diversas. A simula{\c c}{\~a}o de tais casos demanda um esfor{\c c}o significativo na constru{\c c}{\~a}o de um sistema de simula{\c c}{\~a}o, e, traz como benef{\'\i}cio a possibilidade de aplicar esta filosofia em redes de grande porte, tanto de distribui{\c c}{\~a}o quanto de transmiss{\~a}o. Assim, tais sistemas fornecem um complexo e desafiador problema a ser estudado.

Este trabalho abriu uma possibilidade ampla de considerar como as rela{\c c}{\~o}es entre cargas n{\~a}o conectadas eletricamente se influenciam mutuamente. Este, talvez seja um dos aspectos mais importantes acrescentados aqui. Entretanto, pela dificuldade de simula{\c c}{\~a}o desses fen{\^o}menos que surgem na aplica{\c c}{\~a}o pr{\'a}tica, este aspecto n{\~a}o foi devidamente explorado. Trata-se da matriz de correla{\c c}{\~a}o que entra na Equa{\c c}{\~a}o~\ref{eq:f0} e foi citada na Se{\c c}{\~a}o~\ref{subsec:f1} como sendo a matriz de topologia da rede. A simula{\c c}{\~a}o realizada aqui considerou a topologia da rede como {\'u}nica rela{\c c}{\~a}o entre as cargas, ou seja, a conex{\~a}o el{\'e}trica entre estas. Todavia, as conex{\~o}es mec{\^a}nicas ou outras conex{\~o}es f{\'\i}sicas tamb{\'e}m podem fazer com que duas cargas distintas se influenciem. Por exemplo, uma bomba \textit{booster}, que {\'e} utilizada para elevar a press{\~a}o de um fluido antes que este chegue {\`a} suc{\c c}{\~a}o de uma bomba principal, representa uma conex{\c c}{\~a}o puramente mec{\^a}nica, j{\'a} que estas operam, geralmente, em circuitos el{\'e}tricos distintos. Entretanto, em situa{\c c}{\~o}es espec{\'\i}ficas de desligamento inesperado, o sa{\'\i}da da primeira causa baixa press{\~a}o no fluido que chega {\`a} segunda, fazendo com que a mesma tamb{\'e}m desligue inesperadamente. Este importante aspecto pode ser considerado no modelo ao substituir o elemento da linha correspondente {\`a} carga desligada e coluna correspondente {\`a} carga influenciada pelo valor \textit{True}. Assim, uma sugest{\~a}o para continua{\c c}{\~a}o deste trabalho {\'e} a elabora{\c c}{\~a}o de um m{\'o}dulo que opera junto ao supervis{\'o}rio de opera{\c c}{\~a}o do sistema, monitorando o comportamento em tempo real objetivando manter esta matriz atualizada. Assim, o projetista do sistema, ao considerar sua experi{\^e}ncia em conjunto com os operadores, pode inserir uma estimativa inicial para os casos que podem ser previstos enquanto um software se encarrega de atualizar o restante durante a opera{\c c}{\~a}o. Por exemplo, cada vez que uma carga {\'e} desligada, o sistema monitora os segundos seguintes e insere incrementos para cada carga adicional desligada, com valor inversamente proporcional ao tempo decorrido, e decrementos para valores altos sem desligamento concretizado. Para um valor de fronteira definido, o elemento correspondente na matriz de correla{\c c}{\~a}o torna-se \textit{True}. Tal metodologia figura apenas como sugest{\~a}o, devendo seus aspectos e comportamentos serem analisados {\`a} parte.



% --- -----------------------------------------------------------------
% --- Referencias Bibliográficas. (Obrigatório)
% --- -----------------------------------------------------------------
\cleardoublepage
%\bibliographystyle{acm-2} % abbrv - abnt-num - abnt-alf
%\bibliographystyle{uff-ic} % substituido pelo pacote abntex2cite
\bibliography{bibliografia} % arquivo fonte com a bibilografia

% --- -----------------------------------------------------------------
% --- Apêndice.(Opcional)
% --- -----------------------------------------------------------------
\cleardoublepage
\appendix
\chapter{Estrutura do Simulador}
\label{apend:estr}

Conforme descrito no Cap{\'\i}tulo~\ref{cap:impl}, foram utilizadas duas estruturas para este trabalho: um simulador constru{\'\i}do em linguagem C++, utilizando a biblioteca QtC++ para a interface gr{\'a}fica\footnote{Ver Figuras~\ref{fig:sim}, \ref{fig:sim_sub}, \ref{fig:sim_setpar}, \ref{fig:sim_setls}, \ref{fig:sim_pumps}, \ref{fig:sim_panels}, \ref{fig:sim_main}}; e um caderno de notas escrito na linguagem Python 3, atrav{\'e}s da estrutura de interface Jupyter-Notebook. A se{\c c}{\~a}o~\ref{sec:estsim} apresentar{\'a} a estrutura do simulador, enquanto a Se{\c c}{\~a}o~\ref{sec:estana} apresentar{\'a} a da an{\'a}lise.

\section{Simulador} \label{sec:estsim}

O software foi escrito com estrutura de programa{\c c}{\~a}o  orientada a objetos. Assim, al{\'e}m do programa principal (\textit{main}), diversas classes foram criadas para auxiliar a compila{\c c}{\~a}o, bem como a execu{\c c}{\~a}o. A \textbf{API} da biblioteca QtC++ utiliza, por padr{\~a}o, um arquivo com extens{\~a}o `\textit{.ui}', contendo o desenho da interface gr{\'a}fica, um arquivo com extens{\~a}o `\textit{.h}', de cabe{\c c}alho, com as declara{\c c}{\~o}es das classes e fun{\c c}{\~o}es, e um arquivo com extens{\~a}o `\textit{.cpp}', com o c{\'o}digo propriamente dito. A Tabela~\ref{tab:class} apresenta uma lista das classes criadas. Como n{\~a}o foi utilizado nenhum \textit{template} (classe gen{\'e}rica), todos os arquivos de cabe{\c c}alho necessariamente possuem um arquivo de c{\'o}digo associado.

\begin{table}[!h]
	\begin{center}
		\caption{Lista de Classes do Simulador}
		\label{tab:class}
	    \vspace{5pt}
		\begin{tabular}{c c c}
			\hline
			\textbf{Nome} & \textbf{Interface} & \textbf{Cabe{\c c}alho} \\
			\hline\hline
			MainWindow & \textit{mainwindow.ui} & \textit{mainwindow.h}\\
			Clock & \--- & \textit{clock.h} \\
			Panel & \textit{panel.ui} & \textit{panel.h} \\
			Source & \textit{source.ui} & \textit{source.h} \\
			Load & \textit{load.ui} & \textit{load.h} \\
			Settings & \textit{settings.ui} & \textit{settings.h} \\
			Operation & \--- & \textit{operation.h} \\
			Parser & \--- & \textit{parser.h} \\
			LoadShedding & \textit{loadshedding.ui} & \textit{loadshedding.h} \\
			Search & \--- & \textit{search.h} \\
			\hline
		\end{tabular}
	\end{center}
\end{table}

\subsection{Estrutura de Classes} \label{ssec:classes}

A seguir, veremos uma breve descri{\c c}{\~a}o das classes contidas na Tabela~\ref{tab:class}, bem como seus respectivos m{\'e}todos associados.

\subsubsection{MainWindow} \label{sssec:mainwindow}

Aqui temos a janela principal que cont{\'e}m e gerencia todos os elementos do programa, os arquivos abertos, janela principal, elementos gr{\'a}ficos, simula{\c c}{\~a}o, etc. O programa principal (arquivo \textit{main.cpp}) cria uma inst{\^a}ncia desta classe para exibir a janela principal do programa.

A \textbf{API} QtC++ utiliza, como extens{\~a}o da linguagem C++, o conceito de \textit{slots} e \textit{signals}, onde os \textit{slots} s{\~a}o m{\'e}todos ou fun{\c c}{\~o}es que podem ser ativadas quando um sinal conectado a eles {\'e} emitido. Por sua vez, os sinais s{\~a}o definidos e podem ser emitidos na ocorr{\^e}ncia de algum evento, como um clique de mouse ou a execu{\c c}{\~a}o de uma fun{\c c}{\~a}o. Esses conceitos e detalhes podem ser estudados na documenta{\c c}{\~a}o da \textbf{API}. A Tabela~\ref{tab:mainwindow} apresenta a lista de fun{\c c}{\~o}es, \textit{slots} e sinais da classe MainWindow. Esta mesma estrutura de apresenta{\c c}{\~a}o ser{\'a} utilizada nas demais classes. Como trata-se aqui de uma apresenta{\c c}{\~a}o resumida, n{\~a}o ser{\'a} transcrito o c{\'o}digo fonte nem inclu{\'\i}da a estrutura de vari{\'a}veis de cada m{\'e}todo, \textit{slot} ou sinal. O c{\'o}digo fonte original foi desenvolvido e armazenado na plataforma de versionamento de c{\'o}digo GitLab\footnote{\url{https://gitlab.com/noedelima/Power_Load_Shedding}}, atrav{\'e}s de conta privada.

\begin{table}[!h]
    \begin{center}
	    \caption{Lista de M{\'e}todos e Sinais da Classe MainWindow}
	    \label{tab:mainwindow}
	    \vspace{5pt}
		\begin{tabular}{c c c c c}
			\hline
			\textbf{M{\'e}todos} & \textbf{M{\'e}todos} & \textbf{Slots} & \textbf{Slots} & \textbf{Sinais} \\
			\textbf{P{\'u}blicos} & \textbf{Privados} & \textbf{P{\'u}blicos} & \textbf{Privados} & \\
			\hline\hline
			&   & newMainPanel & newSystem & change \\
			&   &   & closeSystem &     \\
			&   &   & openSystem &     \\
			&   &   & save &     \\
			&   &   & saveAs &     \\
			&   &   & setChange &     \\
			&   &   & releaseRunButtons &     \\
			&   &   & setLoadShedding &     \\
			&   &   & setParameters &     \\
			&   &   & setLoadShedFreq &     \\
			&   &   & setLoadShedTime &     \\
			&   &   & setLoadShedPercent &     \\
			&   &   & setConstants &     \\
			&   &   & setTimeScan &     \\
			&   &   & parserBack &     \\
			&   &   & setParser &     \\
			&   &   & setUFLS &     \\
			&   &   & unsetUFLS &     \\
			&   &   & statusLSAct &     \\
			&   &   & statusOverSpeed &     \\
			&   &   & statusUnderSpeed &     \\
			&   &   & checkExit &     \\
			\hline
		\end{tabular}
	\end{center}
\end{table}

\subsubsection{Clock} \label{sssec:clock}

Esta classe utiliza uma fun{\c c}{\~a}o interna para criar uma \textit{thread} a parte, onde esta {\'u}ltima emite pulsos (sinal \textit{scan}) com um intervalo de tempo definido nas configura{\c c}{\~o}es. Estes pulsos, por sua vez, s{\~a}o utilizados em diversas partes do programa para, por exemplo, efetuar uma atualiza{\c c}{\~a}o nos valores de carga e gera{\c c}{\~a}o, interromper uma busca para iniciar uma nova, etc.

A estrutura de fun{\c c}{\~o}es desta classe esta apresentada na Tabela~\ref{tab:clock}.

\begin{table}[!h]
    \begin{center}
	    \caption{Lista de M{\'e}todos e Sinais da Classe Clock}
	    \label{tab:clock}
	    \vspace{5pt}
		\begin{tabular}{c c c c c}
			\hline
			\textbf{M{\'e}todos} & \textbf{M{\'e}todos} & \textbf{Slots} & \textbf{Slots} & \textbf{Sinais} \\
			\textbf{P{\'u}blicos} & \textbf{Privados} & \textbf{P{\'u}blicos} & \textbf{Privados} & \\
			\hline\hline
			run & startWorkInAThread & setPlay &   & scan \\
			&   & setPause &   & pause \\
			&   & setStop &   & finish \\
			&   & setTime &   & newTime \\
			\hline
		\end{tabular}
	\end{center}
\end{table}

\subsubsection{Panel} \label{sssec:panel}

Esta classe cont{\'e}m um arquivo de interface gr{\'a}fica com a representa{\c c}{\~a}o do painel em abas, fornecendo a contru{\c c}{\~a}o dos elementos como barramentos, disjuntores principais dependendo do tipo criado (de uma barra simples ou duas barras interligadas por um disjuntor central e alimenta{\c c}{\~o}es independentes) e {\'a}reas reservadas para inser{\c c}{\~a}o das cargas el{\'e}tricas e, no caso do painel principal, dos geradores.

Os pain{\'e}is de duas barras contam ainda com um intertravamento que impossibilita que o disjuntor central de interliga{\c c}{\~a}o das barras seja ligado quando h{\'a} duas alimenta{\c c}{\~o}es independentes. Trata-se apenas de um intertravamento operacional usual, pois a manobra de alimenta{\c c}{\~a}o unilateral oferece restri{\c c}{\~a}o de carga e s{\'o} deve ser utilizada para manuten{\c c}{\~a}o de um dos disjuntores.

Al{\'e}m dos elementos gr{\'a}ficos, esta classe conte{\'e}m todos os m{\'e}todos necess{\'a}rios associados, conforme listado na Tabela~\ref{tab:panel}.

\begin{table}[!h]
    \begin{center}
	    \caption{Lista de M{\'e}todos e Sinais da Classe Panel}
	    \label{tab:panel}
	    \vspace{5pt}
		\begin{tabular}{c c c c c}
			\hline
			\textbf{M{\'e}todos} & \textbf{M{\'e}todos} & \textbf{Slots} & \textbf{Slots} & \textbf{Sinais} \\
			\textbf{P{\'u}blicos} & \textbf{Privados} & \textbf{P{\'u}blicos} & \textbf{Privados} & \\
			\hline\hline
			getTag &   & readyA & on\_tie\_clicked & busAonline \\
			getBus &   & readyB & on\_feederA\_clicked & busBonline \\
			getPowerA &   & refresh & on\_feederB\_clicked & changeTag \\
			getPowerB &   & enableA & setDemandA & allowA \\
			getNominal &   & enableB & setDemandB & allowB \\
			getLoadsNested &   & addPanel & setTabTag & change \\
			getLoadsA &   & newPanel &   & deleted \\
			getLoadsB &   & addSource &   & newSetPowerA \\
			getSourcesA &   & newSource &   & newSetPowerB \\
			getSourcesB &   & addLoad &   & newPowerA \\
			getPanels &   & newLoad &   & newPowerB \\
			&   & setTag &   & play \\
			&   & setBus &   & pause \\
			&   & settings &   & stop \\
			&   & setPower &   & frequency \\
			&   & bridgePlay &   & trigger \\
			&   & bridgeStop &   & underSpeed \\
			&   & scan &   & overSpeed \\
			&   & setDev &   &     \\
			&   & setAlarm &   &     \\
			&   & triggerLS &   &     \\
			\hline
		\end{tabular}
	\end{center}
\end{table}

\subsubsection{Source} \label{sssec:source}

Esta classe cont{\'e}m todos os elementos necess{\'a}rios para representar os geradores graficamente, efetuar opera{\c c}{\~o}es como ligar e desligar, alterar o nome atribu{\'\i}do a ele, etc.

Para permitir uma melhor representa{\c c}{\~a}o de sistemas reais, os geradores t{\^e}m uma op{\c c}{\~a}o de opera{\c c}{\~a}o em modo manual, permitindo atribuir um valor de pot{\^e}ncia gerada fixo. Este modo permite influenciar manualmente a divis{\~a}o de carga entre os geradores em opera{\c c}{\~a}o, assumindo o controle autom{\'a}tico caso nenhum outro esteja operando deste modo.

A Tabela~\ref{tab:source} apresenta a estrutura de m{\'e}todos desta classe.

\begin{table}[!h]
    \begin{center}
	    \caption{Lista de M{\'e}todos e Sinais da Classe Source}
	    \label{tab:source}
	    \vspace{5pt}
		\begin{tabular}{c c c c c}
			\hline
			\textbf{M{\'e}todos} & \textbf{M{\'e}todos} & \textbf{Slots} & \textbf{Slots} & \textbf{Sinais} \\
			\textbf{P{\'u}blicos} & \textbf{Privados} & \textbf{P{\'u}blicos} & \textbf{Privados} & \\
			\hline\hline
			isOnline &   & ready & on\_onoff\_clicked & status \\
			isAuto &   & play &   & deleted \\
			getDemand &   & stop &   & change \\
			getNominal &   & setOnOff &   &   \\
			getInertia &   & setDemand &   &   \\
			getPoint &   & setPoint &   &   \\
			getTag &   & setTag &   &   \\
			&   & setPower &   &   \\
			&   & setInertia &   &   \\
			&   & settings &   &   \\
			\hline
		\end{tabular}
	\end{center}
\end{table}

\subsubsection{Load} \label{sssec:load}

Esta classe cont{\'e}m todos os elementos necess{\'a}rios para representar graficamente uma carga, al{\'e}m de conter os dados, nomes, ajustes, valores, etc.

A Tabela~\ref{tab:loads} apresenta a estrutura de m{\'e}todos desta classe.

\begin{table}[!h]
    \begin{center}
	    \caption{Lista de M{\'e}todos e Sinais da Classe Loads}
	    \label{tab:loads}
	    \vspace{5pt}
		\begin{tabular}{c c c c c}
			\hline
			\textbf{M{\'e}todos} & \textbf{M{\'e}todos} & \textbf{Slots} & \textbf{Slots} & \textbf{Sinais} \\
			\textbf{P{\'u}blicos} & \textbf{Privados} & \textbf{P{\'u}blicos} & \textbf{Privados} & \\
			\hline\hline
			getTag &   & on\_onoff\_clicked &   & status \\
			getPower &   & ready &   & changeTag \\
			getNominal &   & setTag &   & deleted \\
			getType &   & setSufix &   & change \\
			getWeight &   & allow &   & power \\
			running &   & turn &   &   \\
			&   & settings &   &   \\
			&   & setPower &   &   \\
			&   & setType &   &   \\
			&   & setDemand &   &   \\
			&   & setWeight &   &   \\
			&   & setTagBackground &   &   \\
			\hline
		\end{tabular}
	\end{center}
\end{table}

\subsubsection{Settings} \label{sssec:settings}

Esta classe apresenta uma interface unificada para tornar poss{\'\i}vel realizar todas as configura{\c c}{\~o}es necess{\'a}rias. Embora contenha em si todas as configura{\c c}{\~o}es globais, ela {\'e} acionada sob demanda, exibindo apenas os campos aplic{\'a}veis relativos {\`a} parte do programa que criou a inst{\^a}ncia.

A Tabela~\ref{tab:settings} apresenta a estrutura de m{\'e}todos desta classe, que consiste, basicamente, em \textit{slots} para configurar os valores a serem exibidos (existentes), e sinais com os novos valores alterados ap{\'o}s confirma{\c c}{\~a}o.

\begin{table}[!h]
    \begin{center}
	    \caption{Lista de M{\'e}todos e Sinais da Classe Settings}
	    \label{tab:settings}
	    \vspace{5pt}
		\begin{tabular}{c c c c c}
			\hline
			\textbf{M{\'e}todos} & \textbf{M{\'e}todos} & \textbf{Slots} & \textbf{Slots} & \textbf{Sinais} \\
			\textbf{P{\'u}blicos} & \textbf{Privados} & \textbf{P{\'u}blicos} & \textbf{Privados} & \\
			\hline\hline
			&   & setGlobal & on\_buttonBox\_accepted & power \\
			&   & setLoadShedding & on\_buttonBox\_clicked & type \\
			&   & setPanel &   & inertia \\
			&   & setSource &   & timelap \\
			&   & setLoad &   & LSsets \\
			&   & setTitle &   & LSfreq \\
			&   & setPower &   & LStime \\
			&   & setType &   & weights \\
			&   & setInertia &   &   \\
			&   & setTimelap &   &   \\
			&   & setLSsets &   &   \\
			&   & setLSfreq &   &   \\
			&   & setLStime &   &   \\
			&   & setWeights &   &   \\
			&   & resetValues &   &   \\
			&   & applyValues &   &   \\
			\hline
		\end{tabular}
	\end{center}
\end{table}

\subsubsection{Operation} \label{sssec:operation}

Esta classe contem toda a simula{\c c}{\~a}o num{\'e}rica das cargas e dos geradores, efetuando tamb{\'e}m a atualiza{\c c}{\~a}o da frequ{\^e}ncia do sistema, al{\'e}m de disparar os sinais de alarme e atua{\c c}{\~a}o da rejei{\c c}{\~a}o de carga.

A Tabela~\ref{tab:operation} apresenta a estrutura de m{\'e}todos desta classe que, podemos destacar, inclui muito mais sinais que m{\'e}todos, pois os resultados gerados mexem intrinsecamente em todo o restante do programa.

\begin{table}[!h]
    \begin{center}
	    \caption{Lista de M{\'e}todos e Sinais da Classe Operation}
	    \label{tab:operation}
	    \vspace{5pt}
		\begin{tabular}{c c c c c}
			\hline
			\textbf{M{\'e}todos} & \textbf{M{\'e}todos} & \textbf{Slots} & \textbf{Slots} & \textbf{Sinais} \\
			\textbf{P{\'u}blicos} & \textbf{Privados} & \textbf{P{\'u}blicos} & \textbf{Privados} & \\
			\hline\hline
			updateLoads &   & update &   & load \\
			loadSharing &   &   &   & generation \\
			&   &   &   & maxgeneration \\
			&   &   &   & overload \\
			&   &   &   & frequency \\
			&   &   &   & underSpeed \\
			&   &   &   & overSpeed \\
			\hline
		\end{tabular}
	\end{center}
\end{table}

\subsubsection{Parser} \label{sssec:parser}

O objetivo desta classe {\'e} permitir o salvamento em disco da rede utilizada nas simula{\c c}{\~o}es. Assim, ao criar uma nova estrutura contendo dados de uma rede (real ou fict{\'\i}cia), pode-se salvar todos os dados, tanto dados el{\'e}tricos (topologia, valores nominais, tipos de carga, etc.) quanto as configura{\c c}{\~o}es (constantes de ajuste, pesos de import{\^a}ncia das cargas, ajustes de atua{\c c}{\~a}o do sistema de descarte de carga e tempo de atualiza{\c c}{\~a}o) podem ser salvas em um arquivo para posterior carregamento e utiliza{\c c}{\~a}o. Este arquivo utiliza uma exten{\c c}{\~a}o personalizada para ser associada ao programa (`\textit{arquivo.noah}') e, internamente, utiliza a estrutura \textbf{JSON} com escrita bin{\'a}ria. Esta estrutura permite a cria{\c c}{\~a}o de uma rede complexa ocupando pouco espa{\c c}o em disco.

Esta classe tamb{\'e}m escreve um arquivo em disco no diret{\'o}rio onde est{\'a} salvo o arquivo com a rede utiliz{\'a}da contendo o hist{\'o}rico da simula{\c c}{\~a}o. Este arquivo est{\'a} estruturado no formato \textbf{JSON} textual, sendo de f{\'a}cil importa{\c c}{\~a}o em outras aplica{\c c}{\~o}es de an{\'a}lise, conforme Se{\c c}{\~a}o~\ref{sec:estana}. Este arquino de \textit{log} tem o mesmo nome do arquivo da rede, por{\'e}m, com a extens{\~a}o modificada (`\textit{arquivo.noah.dat}')

A Tabela~\ref{tab:parser} apresenta a estrutura de m{\'e}todos desta classe.

\begin{table}[!h]
    \begin{center}
	    \caption{Lista de M{\'e}todos e Sinais da Classe Parser}
	    \label{tab:parser}
	    \vspace{5pt}
		\begin{tabular}{c c c c c}
			\hline
			\textbf{M{\'e}todos} & \textbf{M{\'e}todos} & \textbf{Slots} & \textbf{Slots} & \textbf{Sinais} \\
			\textbf{P{\'u}blicos} & \textbf{Privados} & \textbf{P{\'u}blicos} & \textbf{Privados} & \\
			\hline\hline
			close & getDataFromFile & clearPath &   & change \\
			open & setDataToFile & writeLog &   &   \\
			saveAs & loadToJson &   &   &   \\
			save & sourceToJson &   &   &   \\
			getFilename & panelToJson &   &   &   \\
			& jsonToLoad &   &   &   \\
			& jsonToSource &   &   &   \\
			& jsonToPanel &   &   &   \\
			\hline
		\end{tabular}
	\end{center}
\end{table}

\subsubsection{LoadShedding} \label{sssec:loadshedding}

Esta classe fornece uma apresenta{\c c}{\~a}o em lista na janela principal contendo a tabela de Rejei{\c c}{\~a}o de carga. Esta mostra n{\~a}o somente a ordem das cargas, como tamb{\'e}m destaca aquelas que devem ser de fato descartadas, agrupando e destacando em escala de cores por etapa, permitindo uma f{\'a}cil visualiza{\c c}{\~a}o da solu{\c c}{\~a}o encontrada em tempo real. {\'E} esta classe quem dispara o ciclo de buscas da classe Search, e, caso detecte algum desbalan{\c c}o de frequ{\^e}ncia, dispara o alarme e posterior desligamento. {\'E}, tamb{\'e}m, respons{\'a}vel por enviar os dados de \textit{log} para a classe Parser escrever no hist{\'o}rico.

A Tabela~\ref{tab:loadshedding} apresenta a estrutura de m{\'e}todos desta classe.

\begin{table}[!h]
    \begin{center}
	    \caption{Lista de M{\'e}todos e Sinais da Classe LoadShedding}
	    \label{tab:loadshedding}
	    \vspace{5pt}
		\begin{tabular}{c c c c c}
			\hline
			\textbf{M{\'e}todos} & \textbf{M{\'e}todos} & \textbf{Slots} & \textbf{Slots} & \textbf{Sinais} \\
			\textbf{P{\'u}blicos} & \textbf{Privados} & \textbf{P{\'u}blicos} & \textbf{Privados} & \\
			\hline\hline
			& updateTable & setLSFreq & setInitialSolution & alarm \\
			&   & setLSTime & setCostInitial & trigger \\
			&   & setLSPercent & setCostBest & update \\
			&   & setLSWeights & debugLog & log \\
			&   & setMainPanel &   &   \\
			&   & updateMainPanel &   &   \\
			&   & updateView &   &   \\
			&   & updateList &   &   \\
			&   & setAlarm &   &   \\
			&   & actLS &   &   \\
			&   & triggerSearch &   &   \\
			\hline
		\end{tabular}
	\end{center}
\end{table}

\subsubsection{Search} \label{sssec:search}

Esta fun{\c c}{\~a}o, que opera como uma classe virtual, {\'e} o n{\'u}cleo operacional deste trabalho, pois {\'e} quem realiza a metaheur{\'\i}stica de busca e retorna o resultado. Apesar de ser uma classe, configura apenas uma interface para receber do programa as informa{\c c}{\~o}es necess{\'a}rias para realizar a busca e retornar os resultados, sendo, na pr{\'a}tica, um m{\'o}dulo. Assim, o programa principal, \textit{mainwindow}, lan{\c c}a uma inst{\^a}ncia informando os dodos operacionais e configura{\c c}{\~o}es, em seguida aciona o m{\'e}todo de busca. A forma como esta classe foi elaborada solicita uma op{\c c}{\~a}o de busca na forma de {\'\i}ndice com par{\^a}metro impl{\'\i}cito. Assim, {\'e} poss{\'\i}vel implementar uma melhoria (\textit{update}) inserindo outros m{\'e}todos de busca al{\'e}m do \textbf{ILS}, permitindo uma compara{\c c}{\~a}o de desempenho.

A Tabela~\ref{tab:search} apresenta a estrutura de m{\'e}todos desta classe.

\begin{table}[!h]
    \begin{center}
	    \caption{Lista de M{\'e}todos e Sinais da Classe Search}
	    \label{tab:search}
	    \vspace{5pt}
		\begin{tabular}{c c c c c}
			\hline
			\textbf{M{\'e}todos} & \textbf{M{\'e}todos} & \textbf{Slots} & \textbf{Slots} & \textbf{Sinais} \\
			\textbf{P{\'u}blicos} & \textbf{Privados} & \textbf{P{\'u}blicos} & \textbf{Privados} & \\
			\hline\hline
			objective &   & play & startWorkInAThread & update \\
			&   & setWeights & setStop & stop \\
			&   & setSolution & searchILS & improve \\
			&   & setPercent & swap & begin \\
			&   & setLoads &   &   \\
			\hline
		\end{tabular}
	\end{center}
\end{table}

\section{An{\'a}lises Estat{\'\i}sticas} \label{sec:estana}

Ap{\'o}s as etapas de simula{\c c}{\~a}o da opera{\c c}{\~a}o em tempo real, os resultados escritos no hist{\'o}rico de opera{\c c}{\~a}o podem ser utilizados para verificar se o comportamento foi satisfat{\'o}rio. Para tanto, a plataforma de desenvolvimento Anaconda\footnote{\url{www.anaconda.com}} fornece um conjunto de ferramentas multiplataforma que utiliza como base a linguagem Python\footnote{\url{www.python.org}}, permitindo mesclar texto e c{\'o}digo, sendo, inclusive, {\'u}til para fins de apresenta{\c c}{\~a}o. Portanto, utilizando a ferramenta integrada Jupyter-Notebook\footnote{\url{www.jupyter.org}}, foi criado, na forma de bloco de notas, um texto base com a explica{\c c}{\~a}o deste trabalho, incluindo a rede de testes utilizada, e uma parte com c{\'o}digos para tratamento. Assim, um arquivo auxiliar, excrito em Python, cont{\'e}m uma classe contendo os m{\'e}todos para importa{\c c}{\~a}o e pr{\'e}-processamento do arquivo de hist{\'o}rico de opera{\c c}{\~a}o, fornecendo os dados obtidos na forma de dicion{\'a}ros e arranjos de sequ{\^e}ncias num{\'e}ricas. Este arquivo auxiliar {\'e} importado dentro do bloco de notas (apenas para ocultar esse volume de c{\'o}digo do arquivo de texto), trazendo os dados prontos para an{\'a}lise.

O tratamento propriamente dito, foi realizado atrav{\'e}s da biblioteca Matplotlib\footnote{\url{www.matplotlib.org}}, que, a partir das sequ{\^e}ncias num{\'e}ricas, forneceu gr{\'a}ficos sequenciais e ordenados dos ganhos obtidos pelas buscas, bem como os valores estat{\'\i}sticos de m{\'e}dia ($\mu$) e desvio padr{\~a}o ($\sigma$). A visualiza{\c c}{\~a}o dos gr{\'a}ficos obtidos a partir dos ganhos ordenados de forma decrescente permitiu inferir, graficamente, que separar os valores acima e abaixo da m{\'e}dia traria uma aproxima{\c c}{\~a}o melhor dos ganhos obtidos a partir de uma solu{\c c}{\~a}o nova em rela{\c c}{\~a}o aos obtidos por simples ajuste de solu{\c c}{\~a}o anterior, aprimorando as solu{\c c}{\~o}es quantitativas e conclus{\c c}{\~o}es qualitativas ao utilizar as m{\'e}dias e desvios separados.

%Como este c{\'o}digo n{\~a}o constitui propriamente um programa sen{\~a}o um m{\'e}todo, n{\~a}o cabe mais detalhes de implementa{\c c}{\~a}o al{\'e}m desta descri{\c c}{\~a}o das ferramentas utilizadas. % Retirado pra nao ficar uma pagina com uma linha solta ao final da dissertacao

\end{document}

